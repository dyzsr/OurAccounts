\documentclass{article}
% generated by Madoko, version 1.1.6
%mdk-data-line={1}


\usepackage[heading-base={2},section-num={False},bib-label={hide},fontspec={True}]{madoko2}
\usepackage[UTF8]{ctex}


\begin{document}



%mdk-data-line={12}
\mdxtitleblockstart{}
%mdk-data-line={12}
\mdxtitle{\mdline{12}移动互联网技术与应用}%mdk

%mdk-data-line={15}
\mdxsubtitle{\mdline{15}大作业报告}%mdk
\mdxauthorstart{}
%mdk-data-line={20}
\mdxauthorname{\mdline{20}姓名:董岩}%mdk

%mdk-data-line={23}
\mdxauthoraddress{\mdline{23}班级:2016211310}%mdk

%mdk-data-line={26}
\mdxauthornote{\mdline{26}学号:2016211225}%mdk
\mdxauthorend\mdxauthorstart{}
%mdk-data-line={31}
\mdxauthorname{\mdline{31}姓名:倪理涵}%mdk

%mdk-data-line={34}
\mdxauthoraddress{\mdline{34}班级:2016211310}%mdk

%mdk-data-line={37}
\mdxauthornote{\mdline{37}学号:2016211156}%mdk
\mdxauthorend\mdtitleauthorrunning{}{}\mdxtitleblockend%mdk

%mdk-data-line={15}
\noindent\mdline{15}\mdbr%mdk

%mdk-data-line={17}
\mdline{17}\mdbr%mdk

%mdk-data-line={19}
\mdline{19}\mdbr%mdk

%mdk-data-line={21}
\mdline{21}~\mdline{21}%mdk

%mdk-data-line={23}
\mdline{23}\mdbr%mdk

%mdk-data-line={25}
\mdline{25}\mdbr%mdk

%mdk-data-line={27}
\mdline{27}\mdbr%mdk

%mdk-data-line={29}
\mdline{29}\mdbr%mdk

%mdk-data-line={32}
\mdhr{}%mdk
\mdline{34}
\begin{mdtoc}%mdk

\section*{Contents}\label{sec-contents}%mdk%mdk

\begin{mdtocblock}%mdk

\mdtocitemx{section}{\mdref{section}{1.\hspace*{0.5em}系统开发的创意与背景}}%mdk

\mdtocitemx{section}{\mdref{section}{2.\hspace*{0.5em}相关技术}}%mdk

\begin{mdtocblock}%mdk

\mdtocitemx{section}{\mdref{section}{2.1.\hspace*{0.5em}开发环境}}%mdk

\mdtocitemx{section}{\mdref{section}{2.2.\hspace*{0.5em}所使用的技术}}%mdk
%mdk
\end{mdtocblock}%mdk

\mdtocitemx{section}{\mdref{section}{3.\hspace*{0.5em}系统功能需求}}%mdk

\begin{mdtocblock}%mdk

\mdtocitemx{section}{\mdref{section}{3.1.\hspace*{0.5em}记账功能}}%mdk

\mdtocitemx{section}{\mdref{section}{3.2.\hspace*{0.5em}统计功能}}%mdk

\mdtocitemx{section}{\mdref{section}{3.3.\hspace*{0.5em}同步功能}}%mdk
%mdk
\end{mdtocblock}%mdk

\mdtocitemx{section}{\mdref{section}{4.\hspace*{0.5em}系统设计与实现}}%mdk

\begin{mdtocblock}%mdk

\mdtocitemx{section}{\mdref{section}{4.1.\hspace*{0.5em}系统总体设计}}%mdk

\mdtocitemx{section}{\mdref{section}{4.2.\hspace*{0.5em}系统物理分布}}%mdk

\mdtocitemx{section}{\mdref{section}{4.3.\hspace*{0.5em}模块设计}}%mdk

\begin{mdtocblock}%mdk

\mdtocitemx{section}{\mdref{section}{4.3.1.\hspace*{0.5em}主界面}}%mdk

\mdtocitemx{section}{\mdref{section}{4.3.2.\hspace*{0.5em}记账模块}}%mdk

\mdtocitemx{section}{\mdref{section}{4.3.3.\hspace*{0.5em}列表统计模块}}%mdk

\mdtocitemx{section}{\mdref{section}{4.3.4.\hspace*{0.5em}图表统计模块}}%mdk

\mdtocitemx{section}{\mdref{section}{4.3.5.\hspace*{0.5em}云端同步模块}}%mdk

\mdtocitemx{section}{\mdref{section}{4.3.6.\hspace*{0.5em}服务器端}}%mdk
%mdk
\end{mdtocblock}%mdk
%mdk
\end{mdtocblock}%mdk

\mdtocitemx{section}{\mdref{section}{5.\hspace*{0.5em}系统可能的拓展}}%mdk

\mdtocitemx{section}{\mdref{section}{6.\hspace*{0.5em}总结体会}}%mdk
%mdk
\end{mdtocblock}%mdk
%mdk
\end{mdtoc}%mdk

%mdk-data-line={36}
\mdhr{}%mdk

%mdk-data-line={38}
\noindent\mdline{38}\mdbr%mdk

%mdk-data-line={40}
\mdline{40}\mdbr%mdk

%mdk-data-line={42}
\mdline{42}\mdbr%mdk

%mdk-data-line={44}
\mdline{44}\mdbr%mdk

%mdk-data-line={47}
\section{\mdline{47}1.\hspace*{0.5em}\mdline{47}系统开发的创意与背景}\label{section}%mdk%mdk

%mdk-data-line={49}
\noindent\mdline{49}当下是移动互联网的时代,手机已是生活的必需品。合理地使用手机可以让生活更加轻松、便捷。
这学期的《移动互联网技术与应用》课程紧跟时代潮流,由来自企业,具有实际工程经验的老师
为我们讲授移动平台软件的开发,以及与服务端的数据交互技术。%mdk

%mdk-data-line={53}
\mdline{53}这一学期,我们通过IOS平台富文本编辑器软件的开发实践,熟悉了IOS平台的软件开发流程;
之前几周进行Android平台的记账软件的开发,是一个比富文本编辑器规模稍大的软件,更使我们熟悉了
移动平台客户端软件的整体结构。%mdk

%mdk-data-line={57}
\mdline{57}这一次的期末大作业,在上一次Android作业的基础上,增加了服务器数据同步的功能,给了我们一个
了解服务器后端框架的机会。%mdk

%mdk-data-line={60}
\section{\mdline{60}2.\hspace*{0.5em}\mdline{60}相关技术}\label{section}%mdk%mdk

%mdk-data-line={62}
\subsection{\mdline{62}2.1.\hspace*{0.5em}\mdline{62}开发环境}\label{section}%mdk%mdk

%mdk-data-line={64}
\begin{itemize}[noitemsep,topsep=\mdcompacttopsep]%mdk

%mdk-data-line={64}
\item\mdline{64}操作系统:Windows \mdline{64}\&\mdline{64} Ubuntu%mdk

%mdk-data-line={65}
\item\mdline{65}工具:Android Studio,VS Code,命令行终端%mdk

%mdk-data-line={66}
\item\mdline{66}语言:Javascript,React(JSX),Golang%mdk
%mdk
\end{itemize}%mdk

%mdk-data-line={68}
\subsection{\mdline{68}2.2.\hspace*{0.5em}\mdline{68}所使用的技术}\label{section}%mdk%mdk

%mdk-data-line={70}
\noindent\mdline{70}我们的应用主要使用React-Native框架开发,开发过程需要借助\mdline{70}\mdcode{React-Native-CLI}\mdline{70},\mdline{70}\mdcode{Android~Studio}\mdline{70}
等工具,在其中涉及到的技术有:%mdk

%mdk-data-line={73}
\begin{itemize}[noitemsep,topsep=\mdcompacttopsep]%mdk

%mdk-data-line={73}
\item\mdline{73}Javascript(ES2016版本)%mdk

%mdk-data-line={74}
\item\mdline{74}NodeJS,基于V8引擎的JS运行环境,通常用于后端和界面的开发%mdk

%mdk-data-line={75}
\item\mdline{75}React,一套用于编写UI的JS类库%mdk

%mdk-data-line={76}
\item\mdline{76}React-Native,一套使用React开发移动端本地应用的框架%mdk

%mdk-data-line={77}
\item\mdline{77}Redux,一个提供可预测程序状态的JS容器%mdk

%mdk-data-line={78}
\item\mdline{78}React-Redux,一套React和Redux相结合的框架%mdk

%mdk-data-line={79}
\item\mdline{79}Gin,一个高性能go语言后端框架%mdk
%mdk
\end{itemize}%mdk

%mdk-data-line={81}
\section{\mdline{81}3.\hspace*{0.5em}\mdline{81}系统功能需求}\label{section}%mdk%mdk

%mdk-data-line={83}
\noindent\mdline{83}这个软件希望做到:%mdk

%mdk-data-line={85}
\begin{itemize}[noitemsep,topsep=\mdcompacttopsep]%mdk

%mdk-data-line={85}
\item\mdline{85}能详尽地记录账目信息%mdk

%mdk-data-line={86}
\item\mdline{86}能统计账目信息%mdk

%mdk-data-line={87}
\item\mdline{87}能以图表的方式查看统计结果%mdk

%mdk-data-line={88}
\item\mdline{88}账目信息能够稳定保存,能在远端保存备份%mdk
%mdk
\end{itemize}%mdk

%mdk-data-line={90}
\noindent\mdline{90}我们将主要的功能需求划分为三个部分:记账功能、统计功能和同步功能。%mdk

%mdk-data-line={92}
\subsection{\mdline{92}3.1.\hspace*{0.5em}\mdline{92}记账功能}\label{section}%mdk%mdk

%mdk-data-line={94}
\noindent\mdline{94}记账功能需要保存具体的账目信息。一条账目具体需要保存:%mdk

%mdk-data-line={96}
\begin{itemize}[noitemsep,topsep=\mdcompacttopsep]%mdk

%mdk-data-line={96}
\item\mdline{96}交易时间(年月日时分)%mdk

%mdk-data-line={97}
\item\mdline{97}交易类型(收入/支出)%mdk

%mdk-data-line={98}
\item\mdline{98}交易项目(购物、餐饮、教育……)%mdk

%mdk-data-line={99}
\item\mdline{99}交易金额%mdk

%mdk-data-line={100}
\item\mdline{100}对这条交易的描述(具体消费了什么……)%mdk

%mdk-data-line={101}
\item\mdline{101}与交易相关的图片(可以有多张)%mdk

%mdk-data-line={102}
\item\mdline{102}交易地点(可以自动获取当前位置)%mdk
%mdk
\end{itemize}%mdk

%mdk-data-line={104}
\noindent\mdline{104}记账功能要有以下用例:%mdk

%mdk-data-line={106}
\begin{itemize}[noitemsep,topsep=\mdcompacttopsep]%mdk

%mdk-data-line={106}
\item\mdline{106}增添新的账单条目

%mdk-data-line={107}
\begin{itemize}[noitemsep,topsep=\mdcompacttopsep]%mdk

%mdk-data-line={107}
\item\mdline{107}新添的账目初始化为默认值%mdk
%mdk
\end{itemize}%mdk%mdk

%mdk-data-line={108}
\item\mdline{108}修改已有账单条目

%mdk-data-line={109}
\begin{itemize}[noitemsep,topsep=\mdcompacttopsep]%mdk

%mdk-data-line={109}
\item\mdline{109}修改交易时间%mdk

%mdk-data-line={110}
\item\mdline{110}修改交易类型%mdk

%mdk-data-line={111}
\item\mdline{111}修改交易项目%mdk

%mdk-data-line={112}
\item\mdline{112}修改交易金额%mdk

%mdk-data-line={113}
\item\mdline{113}修改对交易的描述%mdk

%mdk-data-line={114}
\item\mdline{114}添加或删除交易相关的图片%mdk

%mdk-data-line={115}
\item\mdline{115}修改交易发生的地点%mdk
%mdk
\end{itemize}%mdk%mdk

%mdk-data-line={116}
\item\mdline{116}删除一条账单条目%mdk
%mdk
\end{itemize}%mdk

%mdk-data-line={118}
\subsection{\mdline{118}3.2.\hspace*{0.5em}\mdline{118}统计功能}\label{section}%mdk%mdk

%mdk-data-line={120}
\noindent\mdline{120}统计功能要求能够对一段时间内的交易信息加以整理,然后以较为直观的形式呈现出来。%mdk

%mdk-data-line={122}
\mdline{122}呈现的形式分为两部分,一部分是文字信息,以列表和数字的形式呈现;另一部分是可视化信息,
以图表的形式呈现。%mdk

%mdk-data-line={125}
\mdline{125}统计功能有以下用例:%mdk

%mdk-data-line={127}
\begin{itemize}[noitemsep,topsep=\mdcompacttopsep]%mdk

%mdk-data-line={127}
\item\mdline{127}查看指定月份的收入和支出总额

%mdk-data-line={128}
\begin{itemize}[noitemsep,topsep=\mdcompacttopsep]%mdk

%mdk-data-line={128}
\item\mdline{128}统计一个月的收入总额%mdk

%mdk-data-line={129}
\item\mdline{129}统计一个月的支出总额%mdk
%mdk
\end{itemize}%mdk%mdk

%mdk-data-line={130}
\item\mdline{130}查看某一日的具体交易信息

%mdk-data-line={131}
\begin{itemize}[noitemsep,topsep=\mdcompacttopsep]%mdk

%mdk-data-line={131}
\item\mdline{131}列表显示当日的所有账单%mdk
%mdk
\end{itemize}%mdk%mdk

%mdk-data-line={132}
\item\mdline{132}查看指定月份的支出和收入类别占比

%mdk-data-line={133}
\begin{itemize}[noitemsep,topsep=\mdcompacttopsep]%mdk

%mdk-data-line={133}
\item\mdline{133}图表显示当月支出中各类别的占比%mdk

%mdk-data-line={134}
\item\mdline{134}图表显示当月收入中各类别的占比%mdk
%mdk
\end{itemize}%mdk%mdk
%mdk
\end{itemize}%mdk

%mdk-data-line={136}
\subsection{\mdline{136}3.3.\hspace*{0.5em}\mdline{136}同步功能}\label{section}%mdk%mdk

%mdk-data-line={138}
\noindent\mdline{138}同步功能要求做到账目数据的可持久化保存,数据要在软件停止运行后稳定的存储在手机中。%mdk

%mdk-data-line={140}
\mdline{140}唯一需要保存的数据是所有的账目,因为统计信息可以依据账目信息动态生成,不需要可持久地
存储在手机上。%mdk

%mdk-data-line={143}
\mdline{143}为了支持数据的迁移和备份,软件应有将数据上传至远端服务器和从服务器下载的功能。%mdk

%mdk-data-line={145}
\section{\mdline{145}4.\hspace*{0.5em}\mdline{145}系统设计与实现}\label{section}%mdk%mdk

%mdk-data-line={147}
\subsection{\mdline{147}4.1.\hspace*{0.5em}\mdline{147}系统总体设计}\label{section}%mdk%mdk

%mdk-data-line={149}
\noindent\mdline{149}我们将记账软件在功能上分为四个模块:记账模块、列表统计模块、图表统计模块、和云端同步模块。%mdk

%mdk-data-line={151}
\mdline{151}在软件结构上分为容器和视图两个部分,利用redux框架,视图部分负责显示和交互,容器部分存储软件
数据,通过接收视图传来的信号,对数据进行同步更新。%mdk

%mdk-data-line={154}
\subsection{\mdline{154}4.2.\hspace*{0.5em}\mdline{154}系统物理分布}\label{section}%mdk%mdk

%mdk-data-line={156}
\noindent\mdline{156}系统总体目录结构:%mdk
\begin{mdpre}%mdk
\noindent OurAccounts\\
├──~android\\
│~~~├──~app\\
│~~~├──~build\\
│~~~├──~gradle\\
│~~~├──~keystores\\
│~~~├──~build.gradle\\
│~~~├──~gradle.properties\\
│~~~├──~gradlew\\
│~~~├──~gradlew.bat\\
│~~~├──~local.properties\\
│~~~├──~OurAccounts.iml\\
│~~~└──~settings.gradle\\
├──~node\_modules\\
├──~OurAccountsServer\\
│  ~├──~certs\\
│  ~│  ~├──~server.crt\\
│  ~│  ~└──~server.key\\
│  ~├──~vendor\\
│  ~├──~Gopkg.lock\\
│  ~├──~Gopkg.toml\\
│  ~└──~main.go\\
├──~src\\
│  ~├──~models\\
│  ~│  ~├──~AccountsReducer.js\\
│  ~│  ~├──~MonthsReducer.js\\
│  ~│  ~├──~StatisticsReducer.js\\
│  ~│  ~└──~SyncReducer.js\\
│  ~├──~views\\
│  ~│  ~├──~AccountEdit.js\\
│  ~│  ~├──~Accounts.js\\
│  ~│  ~├──~HomeScreen.js\\
│  ~│  ~├──~Months.js\\
│  ~│  ~├──~MonthsDetail.js\\
│  ~│  ~├──~Signup.js\\
│  ~│  ~├──~Statistics.js\\
│  ~│  ~├──~Sync.js\\
│  ~│  ~├──~User.js\\
│  ~│  ~└──~style.js\\
│  ~├──~App.js\\
│  ~└──~reducer.js\\
├──~\_\_tests\_\_\\
│~~~└──~App-test.js\\
├──~app.json\\
├──~babel.config.js\\
├──~index.js\\
├──~metro.config.js\\
├──~package.json\\
├──~yarn-error.log\\
└──~yarn.lock%mdk
\end{mdpre}\noindent\mdline{211}\mdcode{index.js}\mdline{211}是主体程序的入口,其余的源代码都存放在\mdline{211}\mdcode{src}\mdline{211}文件夹中。\mdline{211}\mdcode{android}\mdline{211}文件夹保存Android
项目信息,主要是React-Native框架自动生成的代码和配置文件。

%mdk-data-line={214}
\mdline{214}\mdcode{src/views}\mdline{214}目录保存各组件的视图界面代码;\mdline{214}\mdcode{src/models}\mdline{214}目录保存各组件数据的容器代码;
\mdline{215}\mdcode{src/reducer.js}\mdline{215}保存Redux框架中程序的总数据容器;\mdline{215}\mdcode{src/App.js}\mdline{215}将数据容器与视图部分结合。%mdk

%mdk-data-line={217}
\subsection{\mdline{217}4.3.\hspace*{0.5em}\mdline{217}模块设计}\label{section}%mdk%mdk

%mdk-data-line={219}
\subsubsection{\mdline{219}4.3.1.\hspace*{0.5em}\mdline{219}主界面}\label{section}%mdk%mdk

%mdk-data-line={221}
\paragraph{\mdline{221}入口}\label{section}%mdk%mdk

%mdk-data-line={223}
\noindent\mdline{223}整个应用的入口是\mdline{223}\mdcode{index.js}\mdline{223}文件,\mdline{223}\mdcode{index.js}\mdline{223}中注册\mdline{223}\mdcode{src/App.js}\mdline{223}中定义的\mdline{223}\mdcode{App}\mdline{223}组件。
\mdline{224}\mdcode{App}\mdline{224}组件是真正的应用程序。%mdk
\begin{mdpre}%mdk
\noindent{\mdcolor{navy}import}~\{AppRegistry\}~{\mdcolor{navy}from}~{\mdcolor{maroon}'}{\mdcolor{maroon}react-native}{\mdcolor{maroon}'};\\
{\mdcolor{navy}import}~App~{\mdcolor{navy}from}~{\mdcolor{maroon}'}{\mdcolor{maroon}./src/App}{\mdcolor{maroon}'};\\
{\mdcolor{navy}import}~\{name~{\mdcolor{navy}as}~appName\}~{\mdcolor{navy}from}~{\mdcolor{maroon}'}{\mdcolor{maroon}./app.json}{\mdcolor{maroon}'};\\
\\
AppRegistry.registerComponent(appName,~()~=\textgreater{}~App);%mdk
\end{mdpre}\noindent\mdline{234}\mdcode{src/App.js}\mdline{234}文件在数据容器与用户界面结合在一起。通过React-Redux框架的\mdline{234}\mdcode{Provider}\mdline{234}组件
向用户界面提供数据。利用\mdline{235}\mdcode{redux-persist}\mdline{235}提供的\mdline{235}\mdcode{PersistDate}\mdline{235}实现账目数据在手机本地的
可持久化存储。\mdline{236}\mdcode{AppContainer}\mdline{236}是主界面组件。\mdline{236}\mdcode{store}\mdline{236}是全局数据存储容器。
\begin{mdpre}%mdk
\noindent{\mdcolor{navy}import}~AppContainer~{\mdcolor{navy}from}~{\mdcolor{maroon}'}{\mdcolor{maroon}./views/HomeScreen}{\mdcolor{maroon}'};\\
{\mdcolor{navy}import}~\{~store,~persistor~\}~{\mdcolor{navy}from}~{\mdcolor{maroon}'}{\mdcolor{maroon}./reducer}{\mdcolor{maroon}'};\\
\\
{\mdcolor{navy}const}~App~=~()~=\textgreater{}~\{\\
~~{\mdcolor{navy}return}~(\\
~~~~\textless{}Provider~store=\{store\}\textgreater{}\\
~~~~~~\textless{}PersistGate~loading=\{\textless{}ActivityIndicator~size={\mdcolor{maroon}"}{\mdcolor{maroon}large}{\mdcolor{maroon}"}/\textgreater{}\}~persistor=\{persistor\}\textgreater{}\\
~~~~~~~~\textless{}AppContainer~/\textgreater{}\\
~~~~~~\textless{}/PersistGate\textgreater{}\\
~~~~\textless{}/Provider\textgreater{}\\
~~);\\
\}\\
\\
{\mdcolor{navy}export}~{\mdcolor{navy}default}~App;%mdk
\end{mdpre}
%mdk-data-line={255}
\paragraph{\mdline{255}数据容器}\label{section}%mdk%mdk

%mdk-data-line={257}
\noindent\mdline{257}\mdcode{/src/reducer.js}\mdline{257}核心代码,创建全局数据容器并做可持久化处理。%mdk
\begin{mdpre}%mdk
\noindent{\mdcolor{navy}const}~reducer~=~combineReducers(\{\\
~~accountInfo:~persistReducer(accountsPersistConfig,~accountsReducer),\\
~~monthInfo:~monthsReducer,\\
~~statisticsInfo:~statisticsReducer,\\
\});\\
\\
{\mdcolor{navy}const}~persistedReducer~=~persistReducer(persistConfig,~reducer);\\
\\
{\mdcolor{navy}const}~store~=~createStore(persistedReducer);\\
{\mdcolor{navy}const}~persistor~=~persistStore(store);%mdk
\end{mdpre}
%mdk-data-line={272}
\paragraph{\mdline{272}主界面}\label{section}%mdk%mdk

%mdk-data-line={274}
\noindent\mdline{274}\mdcode{src/views/HomeScreen}\mdline{274}保存主界面。主界面由标题栏,正文和底部栏构成。%mdk
\begin{mdpre}%mdk
\noindent{\mdcolor{navy}import}~AccountsView~{\mdcolor{navy}from}~{\mdcolor{maroon}'}{\mdcolor{maroon}./Accounts}{\mdcolor{maroon}'};\\
{\mdcolor{navy}import}~MonthsView~{\mdcolor{navy}from}~{\mdcolor{maroon}'}{\mdcolor{maroon}./Months}{\mdcolor{maroon}'};\\
{\mdcolor{navy}import}~StatisticsView~{\mdcolor{navy}from}~{\mdcolor{maroon}'}{\mdcolor{maroon}./Statistics}{\mdcolor{maroon}'};\\
{\mdcolor{navy}import}~SyncView~{\mdcolor{navy}from}~{\mdcolor{maroon}'}{\mdcolor{maroon}./Sync}{\mdcolor{maroon}'};\\
\\
{\mdcolor{darkgreen}/*}{\mdcolor{darkgreen}~创建一个页面导航界面~}{\mdcolor{darkgreen}*/}\\
{\mdcolor{navy}const}~HomeNavigator~=~createBottomTabNavigator(\{\\
~~~~{\mdcolor{darkgreen}/*}{\mdcolor{darkgreen}~主界面的正文~}{\mdcolor{darkgreen}*/}\\
~~accounts:~\{~screen:~~AccountsView~\},~~~~~{\mdcolor{darkgreen}//~记账页面}\\
~~months:~\{~screen:~MonthsView~\},~~~~~~~~~~{\mdcolor{darkgreen}//~列表统计页面}\\
~~statistics:~\{~screen:~StatisticsView~\},~~{\mdcolor{darkgreen}//~图表统计页面}\\
~~sync:~\{~screen:~SyncView~\},~~~~~~~~~~~~~~{\mdcolor{darkgreen}//~同步页面}\\
\},~\{\\
~~~~initialRouteName:~{\mdcolor{maroon}'}{\mdcolor{maroon}accounts}{\mdcolor{maroon}'},\\
~~~~~~~~{\mdcolor{darkgreen}/*}{\mdcolor{darkgreen}~主界面底部栏~}{\mdcolor{darkgreen}*/}\\
~~~~tabBarComponent:~props~=\textgreater{}~\{\\
~~~~~~{\mdcolor{navy}return}~(\\
~~~~~~~~\textless{}Footer\textgreater{}\\
~~~~~~~~~~\textless{}FooterTab\textgreater{}\\
~~~~~~~~~~~~\textless{}Button\textgreater{}\{{\mdcolor{darkgreen}/*}{\mdcolor{darkgreen}~记账页面按钮~}{\mdcolor{darkgreen}*/}\}\textless{}/Button\textgreater{}\\
~~~~~~~~~~~~\textless{}Button\textgreater{}\{{\mdcolor{darkgreen}/*}{\mdcolor{darkgreen}~列表统计页面按钮~}{\mdcolor{darkgreen}*/}\}\textless{}/Button\textgreater{}\\
~~~~~~~~~~~~\textless{}Button\textgreater{}\{{\mdcolor{darkgreen}/*}{\mdcolor{darkgreen}~图表统计页面按钮~}{\mdcolor{darkgreen}*/}\}\textless{}/Button\textgreater{}\\
~~~~~~~~~~~~\textless{}Button\textgreater{}\{{\mdcolor{darkgreen}/*}{\mdcolor{darkgreen}~同步页面按钮~}{\mdcolor{darkgreen}*/}\}\textless{}/Button\textgreater{}\\
~~~~~~~~~~\textless{}/FooterTab\textgreater{}\\
~~~~~~~~\textless{}/Footer\textgreater{}\\
~~~~~~)\\
~~~~\}\\
~~\}\\
);%mdk
\end{mdpre}
%mdk-data-line={309}
\subsubsection{\mdline{309}4.3.2.\hspace*{0.5em}\mdline{309}记账模块}\label{section}%mdk%mdk

%mdk-data-line={311}
\noindent\mdline{311}记账模块提供账单列表界面和账单编辑界面。账单列表界面显示所有的账目,可以在添加或删除一条账目。
账单编辑界面显示一条具体的账目信息,包括交易发生的日期、时间,交易类型、项目、金额、种类,
以及相关图片和交易发生的地址。%mdk

%mdk-data-line={315}
\paragraph{\mdline{315}账单列表}\label{section}%mdk%mdk

%mdk-data-line={317}
\noindent\mdline{317}账单列表部分对应\mdline{317}\mdcode{src/views/Accounts.js}\mdline{317}文件。%mdk

%mdk-data-line={319}
\mdline{319}核心代码如下。\mdline{319}\mdcode{Container}\mdline{319},\mdline{319}\mdcode{Content}\mdline{319},\mdline{319}\mdcode{Button}\mdline{319}均是\mdline{319}\mdcode{native-base}\mdline{319}提供的组件。
\mdline{320}\mdcode{AccountList}\mdline{320}是自定义的列表界面。\mdline{320}\mdcode{Accounts}\mdline{320}会从记账模块的数据容器中获取账目列表数据\mdline{320}\mdcode{accounts}\mdline{320}
和一些可调用的函数如\mdline{321}\mdcode{onClickDel}\mdline{321},并将这些属性传递给子组件使用。%mdk
\begin{mdpre}%mdk
\noindent~~~~~~~~~~~~\textless{}Container\textgreater{}\\
~~~~~~~~~~~~~~~~\textless{}Content\textgreater{}\\
~~~~~~~~~~~~~~~~~~~~{\mdcolor{darkgreen}/*}{\mdcolor{darkgreen}~添加新条目按钮~}{\mdcolor{darkgreen}*/}\\
~~~~~~~~~~~~~~~~~~~~\textless{}Button~onPress=\{...\}\textgreater{}~~{\mdcolor{darkgreen}//~调用回调函数}\\
~~~~~~~~~~~~~~~~~~~~~~~~\textless{}Icon~.../\textgreater{}\\
~~~~~~~~~~~~~~~~~~~~~~~~\textless{}Text\textgreater{}添加\textless{}/Text\textgreater{}\\
~~~~~~~~~~~~~~~~~~~~\textless{}/Button\textgreater{}\\
~~~~~~~~~~~~~~~~~~~~{\mdcolor{darkgreen}/*}{\mdcolor{darkgreen}~账目列表~}{\mdcolor{darkgreen}*/}\\
~~~~~~~~~~~~~~~~~~~~\textless{}AccountList\\
~~~~~~~~~~~~~~~~~~~~~~~~accounts=\{accounts\}~{\mdcolor{darkgreen}//~传递账目数据}\\
~~~~~~~~~~~~~~~~~~~~~~~~onClickDel=\{...\}\\
~~~~~~~~~~~~~~~~~~~~~~~~onClickEdit=\{...\}\\
~~~~~~~~~~~~~~~~~~~~/\textgreater{}\\
~~~~~~~~~~~~~~~~\textless{}/Content\textgreater{}\\
~~~~~~~~~~~~\textless{}/Container\textgreater{}\\
\}%mdk
\end{mdpre}\noindent\mdline{342}定义\mdline{342}\mdcode{AccountList}\mdline{342}组件的代码如下,使用了\mdline{342}\mdcode{react-native}\mdline{342}提供的基本组件\mdline{342}\mdcode{FlatList}\mdline{342}作为列表界面。
\mdline{343}\mdcode{AccountItem}\mdline{343}是自定义的账单条目的视图。\mdline{343}\mdcode{AccountList}\mdline{343}从父组件处接收\mdline{343}\mdcode{accounts}\mdline{343}和回调函数,
将其传给\mdline{344}\mdcode{AccountItem}\mdline{344}。
\begin{mdpre}%mdk
\noindent~~~~~~~~{\mdcolor{darkgreen}/*}{\mdcolor{darkgreen}~对FlatList做封装,得到AccountList组件~}{\mdcolor{darkgreen}*/}\\
~~~~~~~~\textless{}FlatList\\
~~~~~~~~~~~~data=\{accounts\}\\
~~~~~~~~~~~~renderItem=\{(\{item,~index\})~=\textgreater{}~(\\
~~~~~~~~~~~~~~~~{\mdcolor{darkgreen}/*}{\mdcolor{darkgreen}~每一个账单条目~}{\mdcolor{darkgreen}*/}\\
~~~~~~~~~~~~~~~~\textless{}AccountItem\\
~~~~~~~~~~~~~~~~~~~~account=\{item\}~~~~~~~~{\mdcolor{darkgreen}//~传递账单条目}\\
~~~~~~~~~~~~~~~~~~~~index=\{index\}\\
~~~~~~~~~~~~~~~~~~~~onClickDel=\{onClickDel\}\\
~~~~~~~~~~~~~~~~~~~~onClickEdit=\{onClickEdit\}\\
~~~~~~~~~~~~~~~~/\textgreater{}\\
~~~~~~~~~~~~)\}\\
~~~~~~~~/\textgreater{}%mdk
\end{mdpre}\noindent\mdline{362}定义\mdline{362}\mdcode{AccountItem}\mdline{362}的代码如下,\mdline{362}\mdcode{AccountItem}\mdline{362}中定义了列表中应显示的简略账目信息,
以及左滑时出现的删除按钮。\mdline{363}\mdcode{AccountItem}\mdline{363}接收\mdline{363}\mdcode{accounts}\mdline{363}等属性,用于显示和交互。
\begin{mdpre}%mdk
\noindent~~~~~~~~{\mdcolor{darkgreen}/*}{\mdcolor{darkgreen}~将可滑动行做封装,得到AccountItem~}{\mdcolor{darkgreen}*/}\\
~~~~~~~~\textless{}SwipeRow\\
~~~~~~...\\
~~~~~~~~~~~~{\mdcolor{darkgreen}/*}{\mdcolor{darkgreen}~条目简略信息~}{\mdcolor{darkgreen}*/}\\
~~~~~~~~~~~~body=\{\\
~~~~~~~~~~~~~~~~\textless{}Button\\
~~~~~~~~~~...\\
~~~~~~~~~~~~~~~~~~~~onPress=\{()~=\textgreater{}~onClickEdit(index)\}\textgreater{}\\
~~~~~~~~~~~~~~~~~~~~\textless{}H2\textgreater{}\\
~~~~~~~~~~~~~~~~~~~~~~~~条目\{index\}:\\
~~~~~~~~~~~~~~~~~~~~\textless{}/H2\textgreater{}\\
~~~~~~~~~~~~~~~~~~~~\textless{}Text\textgreater{}\\
~~~~~~~~~~~~~~~~~~~~~~~~\{{\mdcolor{darkgreen}/*}{\mdcolor{darkgreen}~显示简略的账目信息~}{\mdcolor{darkgreen}*/}\}\\
~~~~~~~~~~~~~~~~~~~~\textless{}/Text\textgreater{}\\
~~~~~~~~~~~~~~~~\textless{}/Button\textgreater{}\\
~~~~~~~~~~~~\}\\
~~~~~~~~~~~~{\mdcolor{darkgreen}/*}{\mdcolor{darkgreen}~左滑时出现的删除按钮~}{\mdcolor{darkgreen}*/}\\
~~~~~~~~~~~~right=\{\\
~~~~~~~~~~~~~~~~\textless{}Button\\
~~~~~~~~~~...\\
~~~~~~~~~~~~~~~~~~~~onPress=\{()~=\textgreater{}~onClickDel(index)\}\textgreater{}\\
~~~~~~~~~~~~~~~~~~~~\textless{}Icon~active~name={\mdcolor{maroon}'}{\mdcolor{maroon}trash}{\mdcolor{maroon}'}~/\textgreater{}\\
~~~~~~~~~~~~~~~~\textless{}/Button\textgreater{}\\
~~~~~~~~~~~~\}\\
~~~~~~~~/\textgreater{}%mdk
\end{mdpre}
%mdk-data-line={393}
\paragraph{\mdline{393}账单编辑}\label{section}%mdk%mdk

%mdk-data-line={395}
\noindent\mdline{395}账单编辑部分对应\mdline{395}\mdcode{src/views/AccountEdit.js}\mdline{395}文件。%mdk

%mdk-data-line={397}
\mdline{397}核心代码如下,在一个表单里定义了各编辑组件。
\mdline{398}\mdcode{Form}\mdline{398}、\mdline{398}\mdcode{Item}\mdline{398}、\mdline{398}\mdcode{Input}\mdline{398}、\mdline{398}\mdcode{Label}\mdline{398}、\mdline{398}\mdcode{Picker}\mdline{398}、\mdline{398}\mdcode{Image}\mdline{398}等均为\mdline{398}\mdcode{react-native}\mdline{398}或\mdline{398}\mdcode{native-base}\mdline{398}
提供的基本组件。
\mdline{400}\mdcode{MyDatePicker}\mdline{400}、\mdline{400}\mdcode{MyTimePicker}\mdline{400}是自定义的选择日期和时间的组件。
\mdline{401}\mdcode{AccountEdit}\mdline{401}从数据容器中接收属性\mdline{401}\mdcode{accountData}\mdline{401}和一些回调函数。
\mdline{402}\mdcode{accountData}\mdline{402}表示当前修改的账单条目。%mdk
\begin{mdpre}%mdk
\noindent~~~~~~\textless{}Container\textgreater{}\\
~~~~~~~~\textless{}Content\textgreater{}\\
~~~~~~~~~~\textless{}Form\textgreater{}\\
~~~~~~~~~~~~~~~~~~~~~~~~{\mdcolor{darkgreen}/*}{\mdcolor{darkgreen}~选择日期的组件~}{\mdcolor{darkgreen}*/}\\
~~~~~~~~~~~~\textless{}Item~...\textgreater{}\\
~~~~~~~~~~~~~~\textless{}Label\textgreater{}日期\textless{}/Label\textgreater{}\\
~~~~~~~~~~~~~~\textless{}MyDatePicker~...\\
~~~~~~~~~~~~~~~~date=\{accountData.date\}~onChangeDate=\{onChangeDate\}~/\textgreater{}\\
~~~~~~~~~~~~\textless{}/Item\textgreater{}\\
~~~~~~~~~~~~~~~~~~~~~~~~...\\
~~~~~~~~~~~~~~~~~~~~~~~~{\mdcolor{darkgreen}/*}{\mdcolor{darkgreen}~选择账目类型的组件~}{\mdcolor{darkgreen}*/}\\
~~~~~~~~~~~~\textless{}Item~...\textgreater{}\\
~~~~~~~~~~~~~~\textless{}Label\textgreater{}账目类型\textless{}/Label\textgreater{}\\
~~~~~~~~~~~~~~\textless{}Picker\\
~~~~~~~~~~~~~~~~...\\
~~~~~~~~~~~~~~~~selectedValue=\{\\
~~~~~~~~~~~~~~~~~~accountData.isIncome~===~{\mdcolor{maroon}'}{\mdcolor{maroon}undefined}{\mdcolor{maroon}'}~?~{\mdcolor{navy}true}~:~accountData.isIncome\\
~~~~~~~~~~~~~~~~\}\\
~~~~~~~~~~~~~~~~onValueChange=\{(itemValue)~=\textgreater{}~onChangeType(itemValue)\}\textgreater{}\\
~~~~~~~~~~~~~~~~\textless{}Picker.Item~label={\mdcolor{maroon}'}{\mdcolor{maroon}收入}{\mdcolor{maroon}'}~value=\{{\mdcolor{navy}true}\}~/\textgreater{}\\
~~~~~~~~~~~~~~~~\textless{}Picker.Item~label={\mdcolor{maroon}'}{\mdcolor{maroon}支出}{\mdcolor{maroon}'}~value=\{{\mdcolor{navy}false}\}~/\textgreater{}\\
~~~~~~~~~~~~~~\textless{}/Picker\textgreater{}\\
~~~~~~~~~~~~\textless{}/Item\textgreater{}\\
~~~~~~~~~~~~~~~~~~~~~~~~{\mdcolor{darkgreen}/*}{\mdcolor{darkgreen}~选择消费种类(项目)的组件~}{\mdcolor{darkgreen}*/}\\
~~~~~~~~~~~~\textless{}Item~...\textgreater{}\\
~~~~~~~~~~~~~~\textless{}Label\textgreater{}消费种类\textless{}/Label\textgreater{}\\
~~~~~~~~~~~~~~\textless{}Picker\\
~~~~~~~~~~~~~~~~...\\
~~~~~~~~~~~~~~~~selectedValue=\{accountData.item\}\\
~~~~~~~~~~~~~~~~onValueChange=\{(itemValue)~=\textgreater{}~onChangeItem(itemValue)\}\textgreater{}\\
~~~~~~~~~~~~~~~~\textless{}Picker.Item~label={\mdcolor{maroon}'}{\mdcolor{maroon}购物}{\mdcolor{maroon}'}~value=\{{\mdcolor{maroon}'}{\mdcolor{maroon}购物}{\mdcolor{maroon}'}\}~/\textgreater{}\\
~~~~~~~~~~~~~~~~\textless{}Picker.Item~label={\mdcolor{maroon}'}{\mdcolor{maroon}餐饮}{\mdcolor{maroon}'}~value=\{{\mdcolor{maroon}'}{\mdcolor{maroon}餐饮}{\mdcolor{maroon}'}\}~/\textgreater{}\\
~~~~~~~~~~~~~~~~\textless{}Picker.Item~label={\mdcolor{maroon}'}{\mdcolor{maroon}服装}{\mdcolor{maroon}'}~value=\{{\mdcolor{maroon}'}{\mdcolor{maroon}服装}{\mdcolor{maroon}'}\}~/\textgreater{}\\
~~~~~~~~~~~~~~~~\textless{}Picker.Item~label={\mdcolor{maroon}'}{\mdcolor{maroon}生活}{\mdcolor{maroon}'}~value=\{{\mdcolor{maroon}'}{\mdcolor{maroon}生活}{\mdcolor{maroon}'}\}~/\textgreater{}\\
~~~~~~~~~~~~~~~~~~~~~~~~~~~~~~~~...\\
~~~~~~~~~~~~~~~~~~~~~~~~~~~~~~~~...\\
~~~~~~~~~~~~~~\textless{}/Picker\textgreater{}\\
~~~~~~~~~~~~\textless{}/Item\textgreater{}\\
~~~~~~~~~~~~~~~~~~~~~~~~...\\
~~~~~~~~~~~~~~~~~~~~~~~~...\\
~~~~~~~~~~\textless{}/Form\textgreater{}\\
~~~~~~~~\textless{}/Content\textgreater{}\\
~~~~~~\textless{}/Container\textgreater{}%mdk
\end{mdpre}
%mdk-data-line={450}
\paragraph{\mdline{450}账单数据的存储容器}\label{section}%mdk%mdk

%mdk-data-line={452}
\noindent\mdline{452}账单数据的保存应用了Redux框架,即程序中只有唯一的一份数据容器,且只能通过
回调函数间接操作数据内容,不允许在用户界面中对数据直接修改。%mdk

%mdk-data-line={455}
\mdline{455}账目信息对应的数据容器实现在\mdline{455}\mdcode{src/models/AccountReducer.js}\mdline{455}中。%mdk

%mdk-data-line={457}
\mdline{457}定义账单数据类型。%mdk
\begin{mdpre}%mdk
\noindent{\mdcolor{navy}class}~AccountData~\{\\
~~{\mdcolor{navy}constructor}(\{key\})~\{\\
~~~~{\mdcolor{navy}this}.key~=~key;~{\mdcolor{darkgreen}//~string}\\
~~~~{\mdcolor{navy}this}.date~=~moment({\mdcolor{navy}new}~Date()).format({\mdcolor{maroon}'}{\mdcolor{maroon}YYYY-MM-DD}{\mdcolor{maroon}'});~{\mdcolor{darkgreen}//~Date}\\
~~~~{\mdcolor{navy}this}.time~=~moment({\mdcolor{navy}new}~Date()).format({\mdcolor{maroon}"}{\mdcolor{maroon}LT}{\mdcolor{maroon}"});~{\mdcolor{darkgreen}//~Date}\\
~~~~{\mdcolor{navy}this}.isIncome~=~{\mdcolor{navy}false};~{\mdcolor{darkgreen}//~boolean:~is~income~or~expense}\\
~~~~{\mdcolor{navy}this}.amount~=~{\mdcolor{maroon}"}{\mdcolor{maroon}0}{\mdcolor{maroon}"}~{\mdcolor{darkgreen}//~string:~the~amount~of~money}\\
~~~~{\mdcolor{navy}this}.item~=~{\mdcolor{maroon}'}{\mdcolor{maroon}购物}{\mdcolor{maroon}'}~{\mdcolor{darkgreen}//~string:~on~what~item~the~transaction~is}\\
~~~~{\mdcolor{navy}this}.desc~=~undefined~{\mdcolor{darkgreen}//~string:~description~of~the~transaction}\\
~~~~{\mdcolor{navy}this}.imgPaths~=~{}[];~{\mdcolor{darkgreen}//~array(string)~paths~of~images}\\
~~~~{\mdcolor{navy}this}.position~=~undefined;~{\mdcolor{darkgreen}//~Position:~the~geolocation~where~the~transaction~happened}\\
~~\}\\
\}%mdk
\end{mdpre}\noindent\mdline{475}定义默认的容器数据。
\begin{mdpre}%mdk
\noindent{\mdcolor{navy}const}~INITIAL\_STATE~=~\{\\
~~next\_key:~{\mdcolor{purple}0},\\
~~accounts:~{}[],\\
~~index:~{\mdcolor{purple}0},\\
~~accountData:~{\mdcolor{navy}new}~AccountData(\{\}),\\
\};%mdk
\end{mdpre}\noindent\mdline{486}定义所有可能发生的数据操作行为。
\begin{mdpre}%mdk
\noindent{\mdcolor{navy}const}~accountsReducer~=~(state~=~INITIAL\_STATE,~action)~=\textgreater{}~\{\\
~~{\mdcolor{navy}switch}~(action.type)~\{\\
~~~~{\mdcolor{navy}case}~{\mdcolor{maroon}"}{\mdcolor{maroon}account\_add}{\mdcolor{maroon}"}:~{\mdcolor{navy}return}~handleAdd(state);\\
~~~~{\mdcolor{navy}case}~{\mdcolor{maroon}"}{\mdcolor{maroon}account\_del}{\mdcolor{maroon}"}:~{\mdcolor{navy}return}~handleDel(state,~action);\\
~~~~{\mdcolor{navy}case}~{\mdcolor{maroon}"}{\mdcolor{maroon}account\_edit}{\mdcolor{maroon}"}:~{\mdcolor{navy}return}~handleEdit(state,~action);\\
~~~~{\mdcolor{navy}case}~{\mdcolor{maroon}"}{\mdcolor{maroon}account\_save}{\mdcolor{maroon}"}:~{\mdcolor{navy}return}~handleSave(state);\\
~~~~{\mdcolor{navy}case}~{\mdcolor{maroon}"}{\mdcolor{maroon}account\_edit\_date}{\mdcolor{maroon}"}:~{\mdcolor{navy}return}~handleEditDate(state,~action);\\
~~~~{\mdcolor{navy}case}~{\mdcolor{maroon}"}{\mdcolor{maroon}account\_edit\_time}{\mdcolor{maroon}"}:~{\mdcolor{navy}return}~handleEditTime(state,~action);\\
~~~~{\mdcolor{navy}case}~{\mdcolor{maroon}"}{\mdcolor{maroon}account\_edit\_type}{\mdcolor{maroon}"}:~{\mdcolor{navy}return}~handleEditType(state,~action);\\
~~~~{\mdcolor{navy}case}~{\mdcolor{maroon}"}{\mdcolor{maroon}account\_edit\_amount}{\mdcolor{maroon}"}:~{\mdcolor{navy}return}~handleEditAmount(state,~action);\\
~~~~...\\
~~~~~~~~...\\
~~\}\\
~~{\mdcolor{navy}return}~state;\\
\}%mdk
\end{mdpre}
%mdk-data-line={506}
\subsubsection{\mdline{506}4.3.3.\hspace*{0.5em}\mdline{506}列表统计模块}\label{section}%mdk%mdk

%mdk-data-line={508}
\noindent\mdline{508}列表统计模块将每月的总收入和总支出显示给用户。为了方便地查看某一日的具体开支,还要能提供快捷的日期跳转功能。%mdk

%mdk-data-line={510}
\mdline{510}在\mdline{510}\mdcode{react-native-calendars}\mdline{510}中,提供了\mdline{510}\mdcode{Calendar}\mdline{510}组件以显示日历。日历具有点击左右箭头按钮切换月份、单击日期触发事件等功能。%mdk
\begin{mdpre}%mdk
\noindent~~~~~~~~~~~~~~~~\textless{}Calendar\\
~~~~~~~~~~~~~~~~~~~~onDayPress=\{(day)~=\textgreater{}\\
~~~~~~~~~~~~~~~~~~~~~~~~{\mdcolor{darkgreen}//~点击日期切换至消费详细}\\
~~~~~~~~~~~~~~~~~~~~~~~~onClick(day,~()~=\textgreater{}~\{\\
~~~~~~~~~~~~~~~~~~~~~~~~~~~~{\mdcolor{darkgreen}//~console.warn(day);}\\
~~~~~~~~~~~~~~~~~~~~~~~~~~~~navigation.navigate({\mdcolor{maroon}'}{\mdcolor{maroon}monthsDetail}{\mdcolor{maroon}'});\\
~~~~~~~~~~~~~~~~~~~~~~~~\})\\
~~~~~~~~~~~~~~~~~~~~\}\\
~~~~~~~~~~~~~~~~~~~~monthFormat~=~\{~{\mdcolor{maroon}'}{\mdcolor{maroon}yyyy年M月}{\mdcolor{maroon}'}~\}\\
~~~~~~~~~~~~~~~~~~~~onMonthChange~=~\{(month)~=\textgreater{}~\{\\
~~~~~~~~~~~~~~~~~~~~~~~~onChange(month);\\
~~~~~~~~~~~~~~~~~~~~~~~~onIncome(accounts);\\
~~~~~~~~~~~~~~~~~~~~~~~~onExpense(accounts);\\
~~~~~~~~~~~~~~~~~~~~\}\}\\
~~~~~~~~~~~~~~~~/\textgreater{}%mdk
\end{mdpre}\noindent\mdline{530}同时,为了实现UI上的复用,避免多个模块风格不一致,
在\mdline{531}\mdcode{Months}\mdline{531}类型被封装为用户可见的\mdline{531}\mdcode{MonthsView}\mdline{531}前,我们使用\mdline{531}\mdcode{react-navigation}\mdline{531}中的功能指定其“标题栏”的外观。
\begin{mdpre}%mdk
\noindent~~{\mdcolor{navy}static}~navigationOptions(\{navigation\})~\{\\
~~~~~~~~{\mdcolor{navy}return}~\{\\
~~~~~~~~~~~~title:~{\mdcolor{maroon}'}{\mdcolor{maroon}Months}{\mdcolor{maroon}'},\\
~~~~~~~~~~~~header:~(\\
~~~~~~~~~~~~~~~~\textless{}Header\textgreater{}\\
~~~~~~~~~~~~~~~~~~~~\textless{}Left~/\textgreater{}\\
~~~~~~~~~~~~~~~~~~~~\textless{}Body\textgreater{}\\
~~~~~~~~~~~~~~~~~~~~~~~~\textless{}Title\textgreater{}月份\textless{}/Title\textgreater{}\\
~~~~~~~~~~~~~~~~~~~~\textless{}/Body\textgreater{}\\
~~~~~~~~~~~~~~~~~~~~\textless{}Right~/\textgreater{}\\
~~~~~~~~~~~~~~~~\textless{}/Header\textgreater{}\\
~~~~~~~~~~~~)\\
~~~~~~~~\};\\
~~\}%mdk
\end{mdpre}\noindent\mdline{550}在列表统计模块内部的触发事件可分为以下几类。

%mdk-data-line={552}
\begin{itemize}[noitemsep,topsep=\mdcompacttopsep]%mdk

%mdk-data-line={552}
\item\mdline{552}点击日历上的具体日期

%mdk-data-line={553}
\begin{itemize}[noitemsep,topsep=\mdcompacttopsep]%mdk

%mdk-data-line={553}
\item\mdline{553}此时更新状态中的年、月、日,对应到该日期,通过年月日筛选出该日期下的账目%mdk

%mdk-data-line={554}
\item\mdline{554}触发从当前视图\mdline{554}\mdcode{MonthsView\_}\mdline{554}转移到\mdline{554}\mdcode{MonthsDetailView}\mdline{554}的事件%mdk
%mdk
\end{itemize}%mdk%mdk

%mdk-data-line={555}
\item\mdline{555}点击日历的左右切换月份按钮

%mdk-data-line={556}
\begin{itemize}[noitemsep,topsep=\mdcompacttopsep]%mdk

%mdk-data-line={556}
\item\mdline{556}更新状态中的月%mdk

%mdk-data-line={557}
\item\mdline{557}更新当前月的总收入和总支出%mdk
%mdk
\end{itemize}%mdk%mdk
%mdk
\end{itemize}%mdk
\begin{mdpre}%mdk
\noindent~~{\mdcolor{navy}const}~mapDispatchToProps~=~(dispatch)~=\textgreater{}~(\{\\
~~~~~~onClick:~(day,~callBack)~=\textgreater{}~\{\\
~~~~~~~~~~dispatch(\{~type:~{\mdcolor{maroon}'}{\mdcolor{maroon}year\_select}{\mdcolor{maroon}'},~year:~day.year~\});\\
~~~~~~~~~~dispatch(\{~type:~{\mdcolor{maroon}'}{\mdcolor{maroon}month\_select}{\mdcolor{maroon}'},~month:~day.month~\});\\
~~~~~~~~~~dispatch(\{~type:~{\mdcolor{maroon}'}{\mdcolor{maroon}day\_select}{\mdcolor{maroon}'},~day:~day.day~\});\\
~~~~~~~~~~dispatch(\{~type:~{\mdcolor{maroon}'}{\mdcolor{maroon}month\_watch}{\mdcolor{maroon}'},~callBack:~callBack~\})\\
~~~~~~~~~~console.log({\mdcolor{maroon}'}{\mdcolor{maroon}WATCH}{\mdcolor{maroon}'});\\
~~~~~~\},\\
~~~~~~onChange:~(month)~=\textgreater{}~\{\\
~~~~~~~~~~dispatch(\{~type:~{\mdcolor{maroon}'}{\mdcolor{maroon}month\_change}{\mdcolor{maroon}'},~month:~month.month~\});\\
~~~~~~\},\\
~~~~~~onIncome:~(accounts)~=\textgreater{}~\{\\
~~~~~~~~~~dispatch(\{~type:~{\mdcolor{maroon}'}{\mdcolor{maroon}month\_income}{\mdcolor{maroon}'},~accounts:~accounts~\});\\
~~~~~~\},\\
~~~~~~onExpense:~(accounts)~=\textgreater{}~\{\\
~~~~~~~~~~dispatch(\{~type:~{\mdcolor{maroon}'}{\mdcolor{maroon}month\_expense}{\mdcolor{maroon}'},~accounts:~accounts~\});\\
~~~~~~\},\\
~~\});%mdk
\end{mdpre}\noindent\mdline{580}点击某一具体日期后,查看具体账目所涉及的逻辑在\mdline{580}\mdcode{MonthsDetail.js}\mdline{580}中。
在\mdline{581}\mdcode{AccountData}\mdline{581}这一对象类型数据中,\mdline{581}\mdcode{date}\mdline{581}域为形如\mdline{581}\mdcode{"YYYY-MM-DD"}\mdline{581}的字符串。
需要注意的是,诸如\mdline{582}\mdcode{6}\mdline{582}、\mdline{582}\mdcode{7}\mdline{582}月在字符串中也会变为\mdline{582}\mdcode{06}\mdline{582}、\mdline{582}\mdcode{07}\mdline{582}。据此可以写出筛选当日账目并按具体时间排序显示的逻辑。
\begin{mdpre}%mdk
\noindent~~~~~~~~{\mdcolor{navy}var}~res~=~{}[];\\
~~~~~~~~{\mdcolor{navy}for}~({\mdcolor{navy}var}~i~=~{\mdcolor{purple}0};~i~\textless{}~accounts.length;~++~i)~\{\\
~~~~~~~~~~~~{\mdcolor{navy}if}~((month~\textless{}~{\mdcolor{purple}10}~\&\&~year~+~{\mdcolor{maroon}"}{\mdcolor{maroon}-0}{\mdcolor{maroon}"}~+~month~+~{\mdcolor{maroon}"}{\mdcolor{maroon}-}{\mdcolor{maroon}"}~+~day~==~accounts{}[i].date)~\textbar{}\textbar{}\\
~~~~~~~~~~~~(month~\textgreater{}=~{\mdcolor{purple}10}~\&\&~year~+~{\mdcolor{maroon}"}{\mdcolor{maroon}-}{\mdcolor{maroon}"}~+~month~+~{\mdcolor{maroon}"}{\mdcolor{maroon}-}{\mdcolor{maroon}"}~+~day~==~accounts{}[i].date~))~\{\\
~~~~~~~~~~~~~~~~res.push(accounts{}[i]);\\
~~~~~~~~~~~~\}\\
~~~~~~~~\}\\
~~~~~~~~res.sort({\mdcolor{navy}function}(a,~b)\{{\mdcolor{navy}return}~a.time~-~b.time\});%mdk
\end{mdpre}
%mdk-data-line={595}
\subsubsection{\mdline{595}4.3.4.\hspace*{0.5em}\mdline{595}图表统计模块}\label{section}%mdk%mdk

%mdk-data-line={597}
\noindent\mdline{597}我们以饼状图展现每月中,各收入/支出类别在当月总收入/支出的占比。%mdk

%mdk-data-line={599}
\mdline{599}在\mdline{599}\mdcode{react-native-chart-kit}\mdline{599}中,\mdline{599}\mdcode{PieChart}\mdline{599}组件能够满足我们的需求。
在\mdline{600}\mdcode{StatisticsReducer.js}\mdline{600}中,保存的状态用于图表统计模块的制图。
与列表统计模块不同的是,图表统计模块中使用\mdline{601}\mdcode{CalendarList}\mdline{601},无需点按左右箭头按钮切换月份,而是直接左右滑动屏幕即可。%mdk

%mdk-data-line={603}
\mdline{603}状态中,\mdline{603}\mdcode{categories}\mdline{603}中参与统计的收支类型需要与之前记账模块中内置的类型保持一致,\mdline{603}\mdcode{income}\mdline{603}和\mdline{603}\mdcode{expense}\mdline{603}两个数据分别用于收入图和支出图。
但因为所有数据全部为\mdline{604}\mdcode{0}\mdline{604}时,在渲染时会报错,因此我们暂时将“其他”项这里修改为非\mdline{604}\mdcode{0}\mdline{604}。
这只是权宜之计,实际统计后往往会被覆盖掉。%mdk
\begin{mdpre}%mdk
\noindent~~{\mdcolor{navy}const}~INITIAL\_STATE~=~\{\\
~~~~month:~moment().format({\mdcolor{maroon}'}{\mdcolor{maroon}YYYY-MM}{\mdcolor{maroon}'}),\\
~~~~categories:~{}[\\
~~~~~~\{~name:~{\mdcolor{maroon}'}{\mdcolor{maroon}购物}{\mdcolor{maroon}'},~income:~{\mdcolor{purple}0},~expense:~{\mdcolor{purple}0},~\},\\
~~~~~~\{~name:~{\mdcolor{maroon}'}{\mdcolor{maroon}餐饮}{\mdcolor{maroon}'},~income:~{\mdcolor{purple}0},~expense:~{\mdcolor{purple}0},~\},\\
~~~~~~\{~name:~{\mdcolor{maroon}'}{\mdcolor{maroon}服装}{\mdcolor{maroon}'},~income:~{\mdcolor{purple}0},~expense:~{\mdcolor{purple}0},~\},\\
~~~~~~\{~name:~{\mdcolor{maroon}'}{\mdcolor{maroon}生活}{\mdcolor{maroon}'},~income:~{\mdcolor{purple}0},~expense:~{\mdcolor{purple}0},~\},\\
~~~~~~\{~name:~{\mdcolor{maroon}'}{\mdcolor{maroon}教育}{\mdcolor{maroon}'},~income:~{\mdcolor{purple}0},~expense:~{\mdcolor{purple}0},~\},\\
~~~~~~\{~name:~{\mdcolor{maroon}'}{\mdcolor{maroon}娱乐}{\mdcolor{maroon}'},~income:~{\mdcolor{purple}0},~expense:~{\mdcolor{purple}0},~\},\\
~~~~~~\{~name:~{\mdcolor{maroon}'}{\mdcolor{maroon}出行}{\mdcolor{maroon}'},~income:~{\mdcolor{purple}0},~expense:~{\mdcolor{purple}0},~\},\\
~~~~~~\{~name:~{\mdcolor{maroon}'}{\mdcolor{maroon}医疗}{\mdcolor{maroon}'},~income:~{\mdcolor{purple}0},~expense:~{\mdcolor{purple}0},~\},\\
~~~~~~\{~name:~{\mdcolor{maroon}'}{\mdcolor{maroon}投资}{\mdcolor{maroon}'},~income:~{\mdcolor{purple}0},~expense:~{\mdcolor{purple}0},~\},\\
~~~~~~\{~name:~{\mdcolor{maroon}'}{\mdcolor{maroon}其他}{\mdcolor{maroon}'},~income:~{\mdcolor{purple}1},~expense:~{\mdcolor{purple}1},~\},\\
~~~~],\\
~~~~yearData:~{}[],\\
~~\};%mdk
\end{mdpre}\noindent\mdline{626}我们分别对总收入占比和总支出占比进行饼状图制图。
\begin{mdpre}%mdk
\noindent~~~~\textless{}H3~style=\{\{paddingLeft:~{\mdcolor{purple}50},\}\}\textgreater{}月收入统计\textless{}/H3\textgreater{}\\
~~~~\textless{}PieChart\\
~~~~~~~~~~~~...\\
~~~~~~data=\{ctg\_income\}\\
~~~~~~accessor={\mdcolor{maroon}'}{\mdcolor{maroon}population}{\mdcolor{maroon}'}\\
~~~~~~backgroundColor={\mdcolor{maroon}'}{\mdcolor{maroon}transparent}{\mdcolor{maroon}'}\\
~~~~/\textgreater{}\\
\\
~~~~\textless{}H3~...\textgreater{}月支出统计\textless{}/H3\textgreater{}\\
~~~~\textless{}PieChart\\
~~~~~~~~~~~~...\\
~~~~~~data=\{ctg\_expense\}\\
~~~~~~accessor={\mdcolor{maroon}'}{\mdcolor{maroon}population}{\mdcolor{maroon}'}\\
~~~~~~backgroundColor={\mdcolor{maroon}'}{\mdcolor{maroon}transparent}{\mdcolor{maroon}'}\\
~~~~/\textgreater{}%mdk
\end{mdpre}\noindent\mdline{646}最后,在切换月份使得\mdline{646}\mdcode{month}\mdline{646}状态刷新时,我们也要能即时刷新图表显示。这一步的逻辑与列表统计模块类似。
首先更新\mdline{647}\mdcode{month}\mdline{647}属性,然后用更新过的时间属性筛选出对应时间范围内的账目,用它们更新\mdline{647}\mdcode{categories}\mdline{647}中的各条目的收入支出。

%mdk-data-line={649}
\subsubsection{\mdline{649}4.3.5.\hspace*{0.5em}\mdline{649}云端同步模块}\label{section}%mdk%mdk

%mdk-data-line={651}
\noindent\mdline{651}我们的云端同步模块支持上传数据和下载数据功能。%mdk

%mdk-data-line={653}
\mdline{653}同步功能需要用账户来同步数据。用户通过注册界面注册账户,或是通过登陆界面登陆账户。进入账户以后,
可以选择注销,上传数据或者下载数据。%mdk

%mdk-data-line={656}
\mdline{656}在底部的导航栏“我的”项中,首先进入的是登录页面,这一页面提供了两个文本输入框,用来输入用户名和密码,以及两个功能按钮
——在点按登录按钮后,如果用户名和密码正确,则会进入用户的数据同步页面,否则弹出提示框通知用户这一错误。%mdk
\begin{mdpre}%mdk
\noindent\textless{}View\\
~~style~=~\{styles.container\}\\
\textgreater{}\\
~~\{{\mdcolor{darkgreen}/*}{\mdcolor{darkgreen}~账号输入框~}{\mdcolor{darkgreen}*/}\}\\
~~\textless{}View~style=\{{}[styles.view,~styles.lineTopBottom]\}\textgreater{}\\
~~~~\textless{}Text~style=\{styles.text\}\textgreater{}\\
~~~~~~账号\\
~~~~\textless{}/Text\textgreater{}\\
\\
~~~~\textless{}TextInput\\
~~~~~~style=\{styles.textInputStyle\}\\
~~~~~~placeholder={\mdcolor{maroon}"}{\mdcolor{maroon}请输入用户名}{\mdcolor{maroon}"}~~~~{\mdcolor{darkgreen}//~没有任何文字输入时显示}\\
~~~~~~secureTextEntry=\{{\mdcolor{navy}false}\}~~~~{\mdcolor{darkgreen}//~是否敏感}\\
~~~~~~onChangeText=\{onChangeName\}~~~~{\mdcolor{darkgreen}//~文本框内容变化时调用}\\
~~~~~~value=\{name\}\\
~~~~/\textgreater{}\\
~~\textless{}/View\textgreater{}\\
\\
~~\{{\mdcolor{darkgreen}/*}{\mdcolor{darkgreen}~密码输入框~}{\mdcolor{darkgreen}*/}\}\\
~~\textless{}View~style=\{{}[styles.view,~styles.lineTopBottom]\}\textgreater{}\\
~~~~\textless{}Text~style=\{styles.text\}\textgreater{}\\
~~~~~~密码\\
~~~~\textless{}/Text\textgreater{}\\
\\
~~~~\textless{}TextInput\\
~~~~~~style=\{styles.textInputStyle\}\\
~~~~~~placeholder={\mdcolor{maroon}"}{\mdcolor{maroon}请输入密码}{\mdcolor{maroon}"}\\
~~~~~~secureTextEntry=\{{\mdcolor{navy}true}\}\\
~~~~~~onChangeText=\{onChangePswd\}\\
~~~~~~value=\{pswd\}\\
~~~~/\textgreater{}\\
~~\textless{}/View\textgreater{}\\
\textless{}/View\textgreater{}\\
\\
\{{\mdcolor{darkgreen}/*}{\mdcolor{darkgreen}~登录按钮~}{\mdcolor{darkgreen}*/}\}\\
\textless{}View~style=\{styles.buttonView\}\textgreater{}\\
~~\textless{}TouchableOpacity\\
~~~~style=\{styles.button\}\\
~~~~onPress=\{()~=\textgreater{}~signIn(()~=\textgreater{}~navigation.navigate({\mdcolor{maroon}'}{\mdcolor{maroon}userSync}{\mdcolor{maroon}'}))\}~~~~{\mdcolor{darkgreen}//~登录功能}\\
~~\textgreater{}\\
~~~~\textless{}Text~style=\{styles.buttonText\}\textgreater{}登录\textless{}/Text\textgreater{}\\
~~\textless{}/TouchableOpacity\textgreater{}\\
\textless{}/View\textgreater{}\\
\\
\{{\mdcolor{darkgreen}/*}{\mdcolor{darkgreen}~注册按钮~}{\mdcolor{darkgreen}*/}\}\\
\textless{}View~style=\{styles.buttonView\}\textgreater{}\\
~~\textless{}TouchableOpacity\\
~~~~style=\{{}[styles.button,~\{backgroundColor:~{\mdcolor{maroon}"}{\mdcolor{maroon}yellow}{\mdcolor{maroon}"}\}]\}\\
~~~~onPress=\{()~=\textgreater{}~\{navigation.navigate({\mdcolor{maroon}'}{\mdcolor{maroon}signUp}{\mdcolor{maroon}'})\}\}~~~~{\mdcolor{darkgreen}//~跳转至注册}\\
~~\textgreater{}\\
~~~~\textless{}Text~style=\{styles.buttonText\}\textgreater{}注册\textless{}/Text\textgreater{}\\
~~\textless{}/TouchableOpacity\textgreater{}\\
\textless{}/View\textgreater{}%mdk
\end{mdpre}\noindent\mdline{715}还有注册按钮,点击注册按钮后,会进入用户注册页面,包含三个输入框,用来输入用户名,密码以及重复密码。
注册模块会对用户名的合法性以及是否重复,还有两次密码的输入是否一致进行检查。
\begin{mdpre}%mdk
\noindent\textless{}View\\
~~style~=~\{styles.container\}\\
\textgreater{}\\
\\
~~\textless{}View~style=\{styles.inputView\}\textgreater{}\\
~~~~\{{\mdcolor{darkgreen}/*}{\mdcolor{darkgreen}~账号输入框~}{\mdcolor{darkgreen}*/}\}\\
~~~~\textless{}View~style=\{{}[styles.view,~styles.lineTopBottom]\}\textgreater{}\\
~~~~~~\textless{}Text~style=\{styles.text\}\textgreater{}\\
~~~~~~~~账号\\
~~~~~~\textless{}/Text\textgreater{}\\
\\
~~~~~~\textless{}TextInput\\
~~~~~~~~style=\{styles.textInputStyle\}\\
~~~~~~~~placeholder={\mdcolor{maroon}"}{\mdcolor{maroon}6\textasciitilde{}16位,仅包含数字和字母,区分大小写}{\mdcolor{maroon}"}~~~~{\mdcolor{darkgreen}//~没有任何文字输入时显示}\\
~~~~~~~~secureTextEntry=\{{\mdcolor{navy}false}\}~~~~{\mdcolor{darkgreen}//~是否敏感}\\
~~~~~~~~onChangeText=\{onSignName\}~~~~{\mdcolor{darkgreen}//~文本框内容变化时调用}\\
~~~~~~~~value=\{sname\}\\
~~~~~~/\textgreater{}\\
~~~~\textless{}/View\textgreater{}\\
\\
~~~~\{{\mdcolor{darkgreen}/*}{\mdcolor{darkgreen}~密码输入框~}{\mdcolor{darkgreen}*/}\}\\
~~~~\textless{}View~style=\{{}[styles.view,~styles.lineTopBottom]\}\textgreater{}\\
~~~~~~\textless{}Text~style=\{styles.text\}\textgreater{}\\
~~~~~~~~密码\\
~~~~~~\textless{}/Text\textgreater{}\\
\\
~~~~~~\textless{}TextInput\\
~~~~~~~~style=\{styles.textInputStyle\}\\
~~~~~~~~placeholder={\mdcolor{maroon}"}{\mdcolor{maroon}6\textasciitilde{}16位,仅包含数字和字母,区分大小写}{\mdcolor{maroon}"}\\
~~~~~~~~secureTextEntry=\{{\mdcolor{navy}true}\}\\
~~~~~~~~onChangeText=\{onSignPswd\}\\
~~~~~~~~value=\{spswd\}\\
~~~~~~/\textgreater{}\\
~~~~\textless{}/View\textgreater{}\\
\\
~~~~\{{\mdcolor{darkgreen}/*}{\mdcolor{darkgreen}~重复密码输入框~}{\mdcolor{darkgreen}*/}\}\\
~~~~\textless{}View~style=\{{}[styles.view,~styles.lineTopBottom]\}\textgreater{}\\
~~~~~~\textless{}Text~style=\{styles.text\}\textgreater{}\\
~~~~~~~~重复密码\\
~~~~~~\textless{}/Text\textgreater{}\\
\\
~~~~~~\textless{}TextInput\\
~~~~~~~~style=\{styles.textInputStyle\}\\
~~~~~~~~placeholder={\mdcolor{maroon}"}{\mdcolor{maroon}请确认两次密码输入一致}{\mdcolor{maroon}"}\\
~~~~~~~~clearButtonMode={\mdcolor{maroon}"}{\mdcolor{maroon}while-editing}{\mdcolor{maroon}"}\\
~~~~~~~~secureTextEntry=\{{\mdcolor{navy}true}\}\\
~~~~~~~~onChangeText=\{onSignReppswd\}\\
~~~~~~~~value=\{reppswd\}\\
~~~~~~/\textgreater{}\\
~~~~\textless{}/View\textgreater{}\\
~~\textless{}/View\textgreater{}\\
\\
~~\{{\mdcolor{darkgreen}/*}{\mdcolor{darkgreen}~注册按钮~}{\mdcolor{darkgreen}*/}\}\\
~~\textless{}View~style=\{styles.buttonView\}\textgreater{}\\
~~~~\textless{}TouchableOpacity\\
~~~~~~style=\{styles.button\}\\
~~~~~~onPress=\{()~=\textgreater{}~signUp(navigation.goBack)\}~{\mdcolor{darkgreen}//~注册功能}\\
~~~~\textgreater{}\\
~~~~~~\textless{}Text~style=\{styles.buttonText\}\textgreater{}注册\textless{}/Text\textgreater{}\\
~~~~\textless{}/TouchableOpacity\textgreater{}\\
~~\textless{}/View\textgreater{}\\
\\
\textless{}/View\textgreater{}%mdk
\end{mdpre}\noindent\mdline{784}同步模块分为3个页面,因此共有三个文件分别实现UI及功能:\mdline{784}\mdcode{Sync.js}\mdline{784},\mdline{784}\mdcode{User.js}\mdline{784},\mdline{784}\mdcode{Signup.js}\mdline{784}。
以上已经给出了\mdline{785}\mdcode{Sync.js}\mdline{785}和\mdline{785}\mdcode{Signup.js}\mdline{785}的主要逻辑。
而在\mdline{786}\mdcode{User.js}\mdline{786}中,主要实现账目数据的上传与下载,以及注销功能——点按“退出”按钮后,界面会跳转到原先的登录界面。

%mdk-data-line={788}
\mdline{788}需要说明的是,登录、注册、上传以及下载这4大功能,在服务器端制定了相应的API:\mdline{788}\mdcode{signin}\mdline{788}、\mdline{788}\mdcode{signup}\mdline{788}、\mdline{788}\mdcode{syncup}\mdline{788}、\mdline{788}\mdcode{syncdown}\mdline{788}。
在\mdline{789}\mdcode{react-native}\mdline{789}中,我们以\mdline{789}\mdcode{fetch}\mdline{789}来向服务器发送请求,以实现这4大API。%mdk

%mdk-data-line={791}
\mdline{791}所有的请求都是\mdline{791}\mdcode{POST}\mdline{791}类型。请求内部的\mdline{791}\mdcode{body}\mdline{791}是相应参数的\mdline{791}\mdcode{JSON}\mdline{791}格式。%mdk
\begin{mdpre}%mdk
\noindent{\mdcolor{navy}const}~handleSignIn~=~(state,~\{callBack\})~=\textgreater{}~\{\\
~~~~{\mdcolor{navy}if}~(state.name~===~{\mdcolor{maroon}'}{\mdcolor{maroon}'}~\textbar{}\textbar{}~state.pswd~===~{\mdcolor{maroon}'}{\mdcolor{maroon}'})~\{\\
~~~~~~~~Alert.alert({\mdcolor{maroon}'}{\mdcolor{maroon}账号和密码不能为空!}{\mdcolor{maroon}'})\\
~~~~~~~~{\mdcolor{navy}return}~state\\
~~~~\}\\
\\
~~~~fetch({\mdcolor{maroon}'}{\mdcolor{maroon}http://49.234.16.186:60000/signin}{\mdcolor{maroon}'},~\{\\
~~~~~~~~method:~{\mdcolor{maroon}'}{\mdcolor{maroon}POST}{\mdcolor{maroon}'},\\
~~~~~~~~headers:~\{\\
~~~~~~~~~~~~{\mdcolor{maroon}'}{\mdcolor{maroon}Accept}{\mdcolor{maroon}'}:~{\mdcolor{maroon}'}{\mdcolor{maroon}application/json}{\mdcolor{maroon}'},\\
~~~~~~~~~~~~{\mdcolor{maroon}'}{\mdcolor{maroon}Content-Type}{\mdcolor{maroon}'}:~{\mdcolor{maroon}'}{\mdcolor{maroon}application/json}{\mdcolor{maroon}'},\\
~~~~~~~~\},\\
~~~~~~~~body:~JSON.stringify(\{\\
~~~~~~~~~~~~name:~state.name,\\
~~~~~~~~~~~~pswd:~state.pswd,\\
~~~~~~~~\})\\
~~~~\}).then((response)~=\textgreater{}~\{\\
~~~~~~~~console.log({\mdcolor{maroon}"}{\mdcolor{maroon}response:~}{\mdcolor{maroon}"})\\
~~~~~~~~console.log(response)\\
~~~~~~~~console.log({\mdcolor{maroon}"}{\mdcolor{maroon}http~status~code:~}{\mdcolor{maroon}"}~+~response.status)\\
\\
~~~~~~~~{\mdcolor{navy}if}~(response.status~==~{\mdcolor{purple}200})~\{~~~~{\mdcolor{darkgreen}// 账号密码正确,跳转至账户数据同步页面}\\
~~~~~~~~~~~~callBack()\\
~~~~~~~~\}\\
~~~~~~~~{\mdcolor{navy}else}~{\mdcolor{navy}if}~(response.status~==~{\mdcolor{purple}400})~\{~~~~{\mdcolor{darkgreen}//~账号密码错误}\\
~~~~~~~~~~~~Alert.alert({\mdcolor{maroon}'}{\mdcolor{maroon}账号或密码错误!}{\mdcolor{maroon}'})\\
~~~~~~~~~~~~{\mdcolor{navy}return}\\
~~~~~~~~\}\\
~~~~~~~~{\mdcolor{navy}else}~\{~~~~{\mdcolor{darkgreen}//~非程序内置逻辑}\\
~~~~~~~~~~~~Alert.alert({\mdcolor{maroon}'}{\mdcolor{maroon}看到这个说明出~bug~啦!}{\mdcolor{maroon}'})\\
~~~~~~~~\}\\
~~~~\}).{\mdcolor{navy}catch}((error)~=\textgreater{}~\{\\
~~~~~~~~console.error(error)\\
~~~~\})\\
\\
~~~~{\mdcolor{navy}return}~state\\
\}\\
\\
{\mdcolor{navy}const}~handleSignUp~=~(state,~\{callBack\})~=\textgreater{}~\{\\
~~~~{\mdcolor{navy}if}~(state.spswd~!=~state.reppswd)~\{\\
~~~~~~~~Alert.alert({\mdcolor{maroon}'}{\mdcolor{maroon}两次密码输入不一致!}{\mdcolor{maroon}'})\\
~~~~~~~~{\mdcolor{navy}return}~state\\
~~~~\}\\
\\
~~~~{\mdcolor{navy}if}~(state.sname~===~{\mdcolor{maroon}'}{\mdcolor{maroon}'}~\textbar{}\textbar{}~state.spswd~===~{\mdcolor{maroon}'}{\mdcolor{maroon}'})~\{\\
~~~~~~~~Alert.alert({\mdcolor{maroon}'}{\mdcolor{maroon}账号和密码不能为空!}{\mdcolor{maroon}'})\\
~~~~~~~~{\mdcolor{navy}return}~state\\
~~~~\}\\
\\
~~~~fetch({\mdcolor{maroon}'}{\mdcolor{maroon}http://49.234.16.186:60000/signup}{\mdcolor{maroon}'},~\{\\
~~~~~~~~method:~{\mdcolor{maroon}'}{\mdcolor{maroon}POST}{\mdcolor{maroon}'},\\
~~~~~~~~headers:~\{\\
~~~~~~~~~~~~{\mdcolor{maroon}'}{\mdcolor{maroon}Accept}{\mdcolor{maroon}'}:~{\mdcolor{maroon}'}{\mdcolor{maroon}application/json}{\mdcolor{maroon}'},\\
~~~~~~~~~~~~{\mdcolor{maroon}'}{\mdcolor{maroon}Content-Type}{\mdcolor{maroon}'}:~{\mdcolor{maroon}'}{\mdcolor{maroon}application/json}{\mdcolor{maroon}'},\\
~~~~~~~~\},\\
~~~~~~~~body:~JSON.stringify(\{\\
~~~~~~~~~~~~name:~state.sname,\\
~~~~~~~~~~~~pswd:~state.spswd,\\
~~~~~~~~\})\\
~~~~\}).then((response)~=\textgreater{}~\{\\
~~~~~~~~console.log({\mdcolor{maroon}"}{\mdcolor{maroon}response:~}{\mdcolor{maroon}"})\\
~~~~~~~~console.log(response)\\
~~~~~~~~console.log({\mdcolor{maroon}"}{\mdcolor{maroon}http~status~code:~}{\mdcolor{maroon}"}~+~response.status)\\
\\
~~~~~~~~{\mdcolor{navy}if}~(response.status~==~{\mdcolor{purple}200})~\{~~~~{\mdcolor{darkgreen}// 账号密码正确,跳转回登录界面}\\
~~~~~~~~~~~~Alert.alert({\mdcolor{maroon}'}{\mdcolor{maroon}注册成功!}{\mdcolor{maroon}'})\\
~~~~~~~~~~~~callBack()\\
~~~~~~~~~~~~{\mdcolor{navy}return}~\{...state,~sname:~{\mdcolor{maroon}'}{\mdcolor{maroon}'},~spswd:~{\mdcolor{maroon}'}{\mdcolor{maroon}'},~reppswd:~{\mdcolor{maroon}'}{\mdcolor{maroon}'}\}\\
~~~~~~~~\}\\
~~~~~~~~{\mdcolor{navy}else}~{\mdcolor{navy}if}~(response.status~==~{\mdcolor{purple}400})~\{~~~~{\mdcolor{darkgreen}//~账号密码错误}\\
~~~~~~~~~~~~Alert.alert({\mdcolor{maroon}'}{\mdcolor{maroon}账号已存在!}{\mdcolor{maroon}'})\\
~~~~~~~~\}\\
~~~~~~~~{\mdcolor{navy}else}~\{~~~~{\mdcolor{darkgreen}//~非程序内置逻辑}\\
~~~~~~~~~~~~Alert.alert({\mdcolor{maroon}'}{\mdcolor{maroon}看到这个说明出~bug~啦!}{\mdcolor{maroon}'})\\
~~~~~~~~\}\\
~~~~\}).{\mdcolor{navy}catch}((error)~=\textgreater{}~\{\\
~~~~~~~~console.error(error)\\
~~~~\})\\
\\
~~~~{\mdcolor{navy}return}~state\\
\}%mdk
\end{mdpre}\noindent\mdline{877}以数据的上传为例,\mdline{877}\mdcode{body}\mdline{877}中不仅要含有\mdline{877}\mdcode{name}\mdline{877}和\mdline{877}\mdcode{pswd}\mdline{877}以验证身份,还要将当前的账目数据\mdline{877}\mdcode{accounts}\mdline{877}转化为\mdline{877}\mdcode{JSON}\mdline{877}格式作为同步的数据\mdline{877}\mdcode{data}\mdline{877}。
\begin{mdpre}%mdk
\noindent body:~JSON.stringify(\{\\
~~name:~state.name,\\
~~pswd:~state.pswd,\\
~~data:~JSON.stringify(accounts),\\
\})%mdk
\end{mdpre}
%mdk-data-line={887}
\subsubsection{\mdline{887}4.3.6.\hspace*{0.5em}\mdline{887}服务器端}\label{section}%mdk%mdk

%mdk-data-line={889}
\noindent\mdline{889}我们的服务器端为客户端保存数据备份。客户端通过向服务端发送http请求来完成注册、登陆认证和数据同步。%mdk

%mdk-data-line={891}
\mdline{891}我们在服务端使用GO语言作为开发语言,使用Gin来搭建服务端框架。数据库我们采用了MongoDB,
使用mongo-go-driver将go与MongoDB连接。%mdk

%mdk-data-line={894}
\mdline{894}服务端核心代码如下。我们首先建立与MongoDB的连接,然后初始化一个Gin引擎,设置路由。
\mdline{895}\mdcode{/signin}\mdline{895}用作登陆,\mdline{895}\mdcode{/signup}\mdline{895}用作注册,\mdline{895}\mdcode{/syncup}\mdline{895}用作上传数据,\mdline{895}\mdcode{/syncdown}\mdline{895}用作下载数据。%mdk

%mdk-data-line={897}
\mdline{897}具体代码参照\mdline{897}\mdcode{OurAccountsServer/main.go}\mdline{897}文件。%mdk
\begin{mdpre}%mdk
\noindent func~main()~\{\\
\preindent{8}//~Set~up~DB\\
\preindent{8}clientOptions~:=~options.Client().ApplyURI("mongodb://localhost:27017")\\
\\
\preindent{8}client,~err~:=~mongo.Connect(context.TODO(),~clientOptions)\\
\\
\preindent{8}if~err~!=~nil~\{\\
\preindent{16}log.Fatal(err)\\
\preindent{8}\}\\
\\
\preindent{8}//~Check~the~connection\\
\preindent{8}err~=~client.Ping(context.TODO(),~nil)\\
\\
\preindent{8}if~err~!=~nil~\{\\
\preindent{16}log.Fatal(err)\\
\preindent{8}\}\\
\\
\preindent{8}fmt.Println("Connected~to~MongoDB!")\\
\\
\preindent{8}//~Set~up~router\\
\preindent{8}router~:=~gin.Default()\\
\\
\preindent{8}config~:=~cors.DefaultConfig()\\
\preindent{8}config.AllowAllOrigins~=~true\\
\\
\preindent{8}router.Use(cors.New(config))\\
\\
\preindent{8}router.POST("/signin",~func(c~*gin.Context)~\{\\
\preindent{16}handleSignIn(c,~client)\\
\preindent{8}\})\\
\preindent{8}router.POST("/signup",~func(c~*gin.Context)~\{\\
\preindent{16}handleSignUp(c,~client)\\
\preindent{8}\})\\
\preindent{8}router.POST("/syncup",~func(c~*gin.Context)~\{\\
\preindent{16}handleSyncUp(c,~client)\\
\preindent{8}\})\\
\preindent{8}router.POST("/syncdown",~func(c~*gin.Context)~\{\\
\preindent{16}handleSyncDown(c,~client)\\
\preindent{8}\})\\
\\
\preindent{8}go~router.RunTLS(":60001",~"./certs/server.crt",~"./certs/server.key")\\
\preindent{8}router.Run(":60000")\\
\}%mdk
\end{mdpre}
%mdk-data-line={946}
\section{\mdline{946}5.\hspace*{0.5em}\mdline{946}系统可能的拓展}\label{section}%mdk%mdk

%mdk-data-line={948}
\begin{enumerate}%mdk

%mdk-data-line={948}
\item{}
%mdk-data-line={948}
\mdline{948}自动同步功能:支持登陆账户以后自动切换账目内容;支持用户修改账目之后自动上传至云服务器。%mdk%mdk

%mdk-data-line={950}
\item{}
%mdk-data-line={950}
\mdline{950}导出功能:支持将选定日期范围内的账目导出至\mdline{950}\mdcode{.csv}\mdline{950}或\mdline{950}\mdcode{.xls}\mdline{950}格式的文件中,并能方便转移到其他平台。%mdk%mdk

%mdk-data-line={952}
\item{}
%mdk-data-line={952}
\mdline{952}更详细的统计:支持更多维度的统计数据,如年消费项目占比,消费额变化曲线等。%mdk%mdk

%mdk-data-line={954}
\item{}
%mdk-data-line={954}
\mdline{954}设置功能:制定多套显示模式、UI界面等,让用户自定义自己的记账软件。%mdk%mdk
%mdk
\end{enumerate}%mdk

%mdk-data-line={958}
\section{\mdline{958}6.\hspace*{0.5em}\mdline{958}总结体会}\label{section}%mdk%mdk

%mdk-data-line={960}
\noindent\mdline{960}这次综合作业集成了之前的Android作业,加入了Go语言服务器部分的代码,让我们在了解客户端开发的
同时也熟悉了服务端的开发框架。在期末阶段,对于学生来说,完成这些大作业是有着不小的压力,
但是收获也相应的非常巨大。%mdk

%mdk-data-line={964}
\mdline{964}首先锻炼的是压力下高效产出代码的能力。我们小组两人在两周的时间内完成了记账软件的开发,对于
相对还缺乏开发经验的我们来说是个比较令人满意的结果。在这一学期的工程训练之后,不少同学对
项目的开发已经有了一定程度的熟悉和了解,这一点对于未来融入工作环境非常有益。%mdk

%mdk-data-line={968}
\mdline{968}其次是学到了一些新技术。我们在开发过程中对React和Redux框架越来越熟悉。
在实现各种功能时,我们接触了不少React,React-Native框架的组件,深刻地认识到了开源社区的力量。
开源社区相互友好协助的氛围更有利于新技术的传播和改进。%mdk%mdk


\end{document}
