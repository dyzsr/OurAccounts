\documentclass{article}
% generated by Madoko, version 1.1.6
%mdk-data-line={1}


\usepackage[heading-base={2},section-num={False},bib-label={hide},fontspec={True}]{madoko2}
\usepackage[UTF8]{ctex}


\begin{document}



%mdk-data-line={9}
\mdxtitleblockstart{}
%mdk-data-line={9}
\mdxtitle{\mdline{9}移动互联网技术与应用}%mdk

%mdk-data-line={12}
\mdxsubtitle{\mdline{12}Android记账软件开发报告}%mdk
\mdxauthorstart{}
%mdk-data-line={17}
\mdxauthorname{\mdline{17}董岩}%mdk

%mdk-data-line={20}
\mdxauthornote{\mdline{20}2016211225}%mdk
\mdxauthorend\mdxauthorstart{}
%mdk-data-line={25}
\mdxauthorname{\mdline{25}倪理涵}%mdk

%mdk-data-line={28}
\mdxauthornote{\mdline{28}2016211156}%mdk
\mdxauthorend\mdtitleauthorrunning{}{}\mdxtitleblockend%mdk

%mdk-data-line={11}
\mdhr{}%mdk
\mdline{13}
\begin{mdtoc}%mdk

\section*{Contents}\label{sec-contents}%mdk%mdk

\begin{mdtocblock}%mdk

\mdtocitemx{section}{\mdref{section}{1.\hspace*{0.5em}系统开发的创意与背景}}%mdk

\mdtocitemx{section}{\mdref{section}{2.\hspace*{0.5em}相关技术}}%mdk

\begin{mdtocblock}%mdk

\mdtocitemx{section}{\mdref{section}{2.1.\hspace*{0.5em}开发环境}}%mdk

\mdtocitemx{section}{\mdref{section}{2.2.\hspace*{0.5em}所使用的技术}}%mdk
%mdk
\end{mdtocblock}%mdk

\mdtocitemx{section}{\mdref{section}{3.\hspace*{0.5em}系统功能需求}}%mdk

\begin{mdtocblock}%mdk

\mdtocitemx{section}{\mdref{section}{3.1.\hspace*{0.5em}记账功能}}%mdk

\mdtocitemx{section}{\mdref{section}{3.2.\hspace*{0.5em}统计功能}}%mdk

\mdtocitemx{section}{\mdref{section}{3.3.\hspace*{0.5em}同步功能}}%mdk
%mdk
\end{mdtocblock}%mdk

\mdtocitemx{section}{\mdref{section}{4.\hspace*{0.5em}系统设计与实现}}%mdk

\begin{mdtocblock}%mdk

\mdtocitemx{section}{\mdref{section}{4.1.\hspace*{0.5em}系统总体设计}}%mdk

\mdtocitemx{section}{\mdref{section}{4.2.\hspace*{0.5em}系统物理分布}}%mdk

\mdtocitemx{section}{\mdref{section}{4.3.\hspace*{0.5em}模块设计}}%mdk

\begin{mdtocblock}%mdk

\mdtocitemx{section}{\mdref{section}{4.3.1.\hspace*{0.5em}主界面}}%mdk

\mdtocitemx{section}{\mdref{section}{4.3.2.\hspace*{0.5em}记账模块}}%mdk

\mdtocitemx{section}{\mdref{section}{4.3.3.\hspace*{0.5em}列表统计模块}}%mdk

\mdtocitemx{section}{\mdref{section}{4.3.4.\hspace*{0.5em}图表统计模块}}%mdk

\mdtocitemx{section}{\mdref{section}{4.3.5.\hspace*{0.5em}云端同步模块}}%mdk
%mdk
\end{mdtocblock}%mdk
%mdk
\end{mdtocblock}%mdk

\mdtocitemx{section}{\mdref{section}{5.\hspace*{0.5em}系统可能的拓展}}%mdk

\mdtocitemx{section}{\mdref{section}{6.\hspace*{0.5em}总结体会}}%mdk
%mdk
\end{mdtocblock}%mdk
%mdk
\end{mdtoc}%mdk

%mdk-data-line={15}
\mdhr{}%mdk

%mdk-data-line={17}
\section{\mdline{17}1.\hspace*{0.5em}\mdline{17}系统开发的创意与背景}\label{section}%mdk%mdk

%mdk-data-line={19}
\noindent\mdline{19}当下是移动互联网的时代,手机已是生活的必需品。合理地使用手机可以让生活更加轻松、便捷。
这学期的《移动互联网技术与应用》课程紧跟时代潮流,由来自企业,具有实际工程经验的老师
为我们讲授移动平台软件的开发,以及与服务端的数据交互技术。%mdk

%mdk-data-line={23}
\mdline{23}这一学期,我们通过IOS平台富文本编辑器软件的开发实践,熟悉了IOS平台的软件开发流程;
现在进行Android平台的记账软件的开发,是一个比富文本编辑器规模稍大的软件,更是我们熟悉了
移动平台客户端软件的整体结构。%mdk

%mdk-data-line={27}
\section{\mdline{27}2.\hspace*{0.5em}\mdline{27}相关技术}\label{section}%mdk%mdk

%mdk-data-line={29}
\subsection{\mdline{29}2.1.\hspace*{0.5em}\mdline{29}开发环境}\label{section}%mdk%mdk

%mdk-data-line={31}
\begin{itemize}[noitemsep,topsep=\mdcompacttopsep]%mdk

%mdk-data-line={31}
\item\mdline{31}操作系统:Windows 10,Ubuntu 18.04 LTS%mdk

%mdk-data-line={32}
\item\mdline{32}工具:Android Studio,VS Code,命令行终端%mdk

%mdk-data-line={33}
\item\mdline{33}语言:Javascript,React(JSX)%mdk
%mdk
\end{itemize}%mdk

%mdk-data-line={35}
\subsection{\mdline{35}2.2.\hspace*{0.5em}\mdline{35}所使用的技术}\label{section}%mdk%mdk

%mdk-data-line={37}
\noindent\mdline{37}我们的应用主要使用React-Native框架开发,开发过程需要借助\mdline{37}\mdcode{React-Native-CLI}\mdline{37},\mdline{37}\mdcode{Android~Studio}\mdline{37}
等工具,在其中涉及到的技术有:%mdk

%mdk-data-line={40}
\begin{itemize}[noitemsep,topsep=\mdcompacttopsep]%mdk

%mdk-data-line={40}
\item\mdline{40}Javascript(ES2016版本)%mdk

%mdk-data-line={41}
\item\mdline{41}NodeJS,基于V8引擎的JS运行环境,通常用于后端和界面的开发%mdk

%mdk-data-line={42}
\item\mdline{42}React,一套用于编写UI的JS类库%mdk

%mdk-data-line={43}
\item\mdline{43}React-Native,一套使用React开发移动端本地应用的框架%mdk

%mdk-data-line={44}
\item\mdline{44}Redux,一个提供可预测程序状态的JS容器%mdk

%mdk-data-line={45}
\item\mdline{45}React-Redux,一套React和Redux相结合的框架%mdk
%mdk
\end{itemize}%mdk

%mdk-data-line={47}
\section{\mdline{47}3.\hspace*{0.5em}\mdline{47}系统功能需求}\label{section}%mdk%mdk

%mdk-data-line={49}
\noindent\mdline{49}这个软件希望做到:%mdk

%mdk-data-line={51}
\begin{itemize}[noitemsep,topsep=\mdcompacttopsep]%mdk

%mdk-data-line={51}
\item\mdline{51}能详尽地记录账目信息%mdk

%mdk-data-line={52}
\item\mdline{52}能统计账目信息%mdk

%mdk-data-line={53}
\item\mdline{53}能以图表的方式查看统计结果%mdk

%mdk-data-line={54}
\item\mdline{54}账目信息能够稳定保存%mdk
%mdk
\end{itemize}%mdk

%mdk-data-line={56}
\noindent\mdline{56}我们将主要的功能需求划分为三个部分:记账功能、统计功能和同步功能。%mdk

%mdk-data-line={58}
\subsection{\mdline{58}3.1.\hspace*{0.5em}\mdline{58}记账功能}\label{section}%mdk%mdk

%mdk-data-line={60}
\noindent\mdline{60}记账功能需要保存具体的账目信息。一条账目具体需要保存:%mdk

%mdk-data-line={62}
\begin{itemize}[noitemsep,topsep=\mdcompacttopsep]%mdk

%mdk-data-line={62}
\item\mdline{62}交易时间(年月日时分)%mdk

%mdk-data-line={63}
\item\mdline{63}交易类型(收入/支出)%mdk

%mdk-data-line={64}
\item\mdline{64}交易项目(购物、餐饮、教育……)%mdk

%mdk-data-line={65}
\item\mdline{65}交易金额%mdk

%mdk-data-line={66}
\item\mdline{66}对这条交易的描述(具体消费了什么……)%mdk

%mdk-data-line={67}
\item\mdline{67}与交易相关的图片(可以有多张)%mdk

%mdk-data-line={68}
\item\mdline{68}交易地点(可以自动获取当前位置)%mdk
%mdk
\end{itemize}%mdk

%mdk-data-line={70}
\noindent\mdline{70}记账功能要有以下用例:%mdk

%mdk-data-line={72}
\begin{itemize}[noitemsep,topsep=\mdcompacttopsep]%mdk

%mdk-data-line={72}
\item\mdline{72}增添新的账单条目

%mdk-data-line={73}
\begin{itemize}[noitemsep,topsep=\mdcompacttopsep]%mdk

%mdk-data-line={73}
\item\mdline{73}新添的账目初始化为默认值%mdk
%mdk
\end{itemize}%mdk%mdk

%mdk-data-line={74}
\item\mdline{74}修改已有账单条目

%mdk-data-line={75}
\begin{itemize}[noitemsep,topsep=\mdcompacttopsep]%mdk

%mdk-data-line={75}
\item\mdline{75}修改交易时间%mdk

%mdk-data-line={76}
\item\mdline{76}修改交易类型%mdk

%mdk-data-line={77}
\item\mdline{77}修改交易项目%mdk

%mdk-data-line={78}
\item\mdline{78}修改交易金额%mdk

%mdk-data-line={79}
\item\mdline{79}修改对交易的描述%mdk

%mdk-data-line={80}
\item\mdline{80}添加或删除交易相关的图片%mdk

%mdk-data-line={81}
\item\mdline{81}修改交易发生的地点%mdk
%mdk
\end{itemize}%mdk%mdk

%mdk-data-line={82}
\item\mdline{82}删除一条账单条目%mdk
%mdk
\end{itemize}%mdk

%mdk-data-line={84}
\subsection{\mdline{84}3.2.\hspace*{0.5em}\mdline{84}统计功能}\label{section}%mdk%mdk

%mdk-data-line={86}
\noindent\mdline{86}统计功能要求能够对一段时间内的交易信息加以整理,然后以较为直观的形式呈现出来。%mdk

%mdk-data-line={88}
\mdline{88}呈现的形式分为两部分,一部分是文字信息,以列表和数字的形式呈现;另一部分是可视化信息,
以图表的形式呈现。%mdk

%mdk-data-line={91}
\mdline{91}统计功能有以下用例:%mdk

%mdk-data-line={93}
\begin{itemize}[noitemsep,topsep=\mdcompacttopsep]%mdk

%mdk-data-line={93}
\item\mdline{93}查看指定月份的收入和支出总额

%mdk-data-line={94}
\begin{itemize}[noitemsep,topsep=\mdcompacttopsep]%mdk

%mdk-data-line={94}
\item\mdline{94}统计一个月的收入总额%mdk

%mdk-data-line={95}
\item\mdline{95}统计一个月的支出总额%mdk
%mdk
\end{itemize}%mdk%mdk

%mdk-data-line={96}
\item\mdline{96}查看某一日的具体交易信息

%mdk-data-line={97}
\begin{itemize}[noitemsep,topsep=\mdcompacttopsep]%mdk

%mdk-data-line={97}
\item\mdline{97}列表显示当日的所有账单%mdk
%mdk
\end{itemize}%mdk%mdk

%mdk-data-line={98}
\item\mdline{98}查看指定月份的支出和收入类别占比

%mdk-data-line={99}
\begin{itemize}[noitemsep,topsep=\mdcompacttopsep]%mdk

%mdk-data-line={99}
\item\mdline{99}图表显示当月支出中各类别的占比%mdk

%mdk-data-line={100}
\item\mdline{100}图表显示当月收入中各类别的占比%mdk
%mdk
\end{itemize}%mdk%mdk
%mdk
\end{itemize}%mdk

%mdk-data-line={102}
\subsection{\mdline{102}3.3.\hspace*{0.5em}\mdline{102}同步功能}\label{section}%mdk%mdk

%mdk-data-line={104}
\noindent\mdline{104}同步功能要求做到账目数据的可持久化保存,数据要在软件停止运行后稳定的存储在手机中。%mdk

%mdk-data-line={106}
\mdline{106}唯一需要保存的数据是所有的账目,因为统计信息可以依据账目信息动态生成,不需要可持久地
存储在手机上。%mdk

%mdk-data-line={109}
\section{\mdline{109}4.\hspace*{0.5em}\mdline{109}系统设计与实现}\label{section}%mdk%mdk

%mdk-data-line={111}
\subsection{\mdline{111}4.1.\hspace*{0.5em}\mdline{111}系统总体设计}\label{section}%mdk%mdk

%mdk-data-line={113}
\noindent\mdline{113}我们将记账软件在功能上分为四个模块:记账模块、列表统计模块、图表统计模块、和云端同步模块。%mdk

%mdk-data-line={115}
\mdline{115}在软件结构上分为容器和视图两个部分,利用redux框架,视图部分负责显示和交互,容器部分存储软件
数据,通过接收视图传来的信号,对数据进行同步更新。%mdk

%mdk-data-line={118}
\subsection{\mdline{118}4.2.\hspace*{0.5em}\mdline{118}系统物理分布}\label{section}%mdk%mdk

%mdk-data-line={120}
\noindent\mdline{120}系统总体目录结构:%mdk
\begin{mdpre}%mdk
\noindent OurAccounts\\
├──~android\\
│~~~├──~app\\
│~~~├──~build\\
│~~~├──~gradle\\
│~~~├──~keystores\\
│~~~├──~build.gradle\\
│~~~├──~gradle.properties\\
│~~~├──~gradlew\\
│~~~├──~gradlew.bat\\
│~~~├──~local.properties\\
│~~~├──~OurAccounts.iml\\
│~~~└──~settings.gradle\\
├──~node\_modules\\
├──~src\\
│~~~├──~models\\
│~~~├──~views\\
│~~~├──~App.js\\
│~~~└──~reducer.js\\
├──~\_\_tests\_\_\\
│~~~└──~App-test.js\\
├──~app.json\\
├──~babel.config.js\\
├──~index.js\\
├──~metro.config.js\\
├──~package.json\\
├──~yarn-error.log\\
└──~yarn.lock%mdk
\end{mdpre}\noindent\mdline{153}\mdcode{index.js}\mdline{153}是主体程序的入口,其余的源代码都存放在\mdline{153}\mdcode{src}\mdline{153}文件夹中。\mdline{153}\mdcode{android}\mdline{153}文件夹保存Android
项目信息,主要是React-Native框架自动生成的代码和配置文件。

%mdk-data-line={156}
\mdline{156}\mdcode{src/views}\mdline{156}目录保存各组件的视图界面代码;\mdline{156}\mdcode{src/models}\mdline{156}目录保存各组件数据的容器代码;
\mdline{157}\mdcode{src/reducer.js}\mdline{157}保存Redux框架中程序的总数据容器;\mdline{157}\mdcode{src/App.js}\mdline{157}将数据容器与视图部分结合。%mdk

%mdk-data-line={159}
\subsection{\mdline{159}4.3.\hspace*{0.5em}\mdline{159}模块设计}\label{section}%mdk%mdk

%mdk-data-line={161}
\subsubsection{\mdline{161}4.3.1.\hspace*{0.5em}\mdline{161}主界面}\label{section}%mdk%mdk

%mdk-data-line={163}
\paragraph{\mdline{163}入口}\label{section}%mdk%mdk

%mdk-data-line={165}
\noindent\mdline{165}整个应用的入口是\mdline{165}\mdcode{index.js}\mdline{165}文件,\mdline{165}\mdcode{index.js}\mdline{165}中注册\mdline{165}\mdcode{src/App.js}\mdline{165}中定义的\mdline{165}\mdcode{App}\mdline{165}组件。
\mdline{166}\mdcode{App}\mdline{166}组件是真正的应用程序。%mdk
\begin{mdpre}%mdk
\noindent{\mdcolor{navy}import}~\{AppRegistry\}~{\mdcolor{navy}from}~{\mdcolor{maroon}'}{\mdcolor{maroon}react-native}{\mdcolor{maroon}'};\\
{\mdcolor{navy}import}~App~{\mdcolor{navy}from}~{\mdcolor{maroon}'}{\mdcolor{maroon}./src/App}{\mdcolor{maroon}'};\\
{\mdcolor{navy}import}~\{name~{\mdcolor{navy}as}~appName\}~{\mdcolor{navy}from}~{\mdcolor{maroon}'}{\mdcolor{maroon}./app.json}{\mdcolor{maroon}'};\\
\\
AppRegistry.registerComponent(appName,~()~=\textgreater{}~App);%mdk
\end{mdpre}\noindent\mdline{176}\mdcode{src/App.js}\mdline{176}文件在数据容器与用户界面结合在一起。通过React-Redux框架的\mdline{176}\mdcode{Provider}\mdline{176}组件
向用户界面提供数据。利用\mdline{177}\mdcode{redux-persist}\mdline{177}提供的\mdline{177}\mdcode{PersistDate}\mdline{177}实现账目数据在手机本地的
可持久化存储。\mdline{178}\mdcode{AppContainer}\mdline{178}是主界面组件。\mdline{178}\mdcode{store}\mdline{178}是全局数据存储容器。
\begin{mdpre}%mdk
\noindent{\mdcolor{navy}import}~AppContainer~{\mdcolor{navy}from}~{\mdcolor{maroon}'}{\mdcolor{maroon}./views/HomeScreen}{\mdcolor{maroon}'};\\
{\mdcolor{navy}import}~\{~store,~persistor~\}~{\mdcolor{navy}from}~{\mdcolor{maroon}'}{\mdcolor{maroon}./reducer}{\mdcolor{maroon}'};\\
\\
{\mdcolor{navy}const}~App~=~()~=\textgreater{}~\{\\
~~{\mdcolor{navy}return}~(\\
~~~~\textless{}Provider~store=\{store\}\textgreater{}\\
~~~~~~\textless{}PersistGate~loading=\{\textless{}ActivityIndicator~size={\mdcolor{maroon}"}{\mdcolor{maroon}large}{\mdcolor{maroon}"}/\textgreater{}\}~persistor=\{persistor\}\textgreater{}\\
~~~~~~~~\textless{}AppContainer~/\textgreater{}\\
~~~~~~\textless{}/PersistGate\textgreater{}\\
~~~~\textless{}/Provider\textgreater{}\\
~~);\\
\}\\
\\
{\mdcolor{navy}export}~{\mdcolor{navy}default}~App;%mdk
\end{mdpre}
%mdk-data-line={197}
\paragraph{\mdline{197}数据容器}\label{section}%mdk%mdk

%mdk-data-line={199}
\noindent\mdline{199}\mdcode{/src/reducer.js}\mdline{199}核心代码,创建全局数据容器并做可持久化处理。%mdk
\begin{mdpre}%mdk
\noindent{\mdcolor{navy}const}~reducer~=~combineReducers(\{\\
~~accountInfo:~persistReducer(accountsPersistConfig,~accountsReducer),\\
~~monthInfo:~monthsReducer,\\
~~statisticsInfo:~statisticsReducer,\\
\});\\
\\
{\mdcolor{navy}const}~persistedReducer~=~persistReducer(persistConfig,~reducer);\\
\\
{\mdcolor{navy}const}~store~=~createStore(persistedReducer);\\
{\mdcolor{navy}const}~persistor~=~persistStore(store);%mdk
\end{mdpre}
%mdk-data-line={214}
\paragraph{\mdline{214}主界面}\label{section}%mdk%mdk

%mdk-data-line={216}
\noindent\mdline{216}\mdcode{src/views/HomeScreen}\mdline{216}保存主界面。主界面由标题栏,正文和底部栏构成。%mdk
\begin{mdpre}%mdk
\noindent{\mdcolor{navy}import}~AccountsView~{\mdcolor{navy}from}~{\mdcolor{maroon}'}{\mdcolor{maroon}./Accounts}{\mdcolor{maroon}'};\\
{\mdcolor{navy}import}~MonthsView~{\mdcolor{navy}from}~{\mdcolor{maroon}'}{\mdcolor{maroon}./Months}{\mdcolor{maroon}'};\\
{\mdcolor{navy}import}~StatisticsView~{\mdcolor{navy}from}~{\mdcolor{maroon}'}{\mdcolor{maroon}./Statistics}{\mdcolor{maroon}'};\\
{\mdcolor{navy}import}~SyncView~{\mdcolor{navy}from}~{\mdcolor{maroon}'}{\mdcolor{maroon}./Sync}{\mdcolor{maroon}'};\\
\\
{\mdcolor{darkgreen}/*}{\mdcolor{darkgreen}~创建一个页面导航界面~}{\mdcolor{darkgreen}*/}\\
{\mdcolor{navy}const}~HomeNavigator~=~createBottomTabNavigator(\{\\
~~~~{\mdcolor{darkgreen}/*}{\mdcolor{darkgreen}~主界面的正文~}{\mdcolor{darkgreen}*/}\\
~~accounts:~\{~screen:~~AccountsView~\},~~~~~{\mdcolor{darkgreen}//~记账页面}\\
~~months:~\{~screen:~MonthsView~\},~~~~~~~~~~{\mdcolor{darkgreen}//~列表统计页面}\\
~~statistics:~\{~screen:~StatisticsView~\},~~{\mdcolor{darkgreen}//~图表统计页面}\\
~~sync:~\{~screen:~SyncView~\},~~~~~~~~~~~~~~{\mdcolor{darkgreen}//~同步页面}\\
\},~\{\\
~~~~initialRouteName:~{\mdcolor{maroon}'}{\mdcolor{maroon}accounts}{\mdcolor{maroon}'},\\
~~~~~~~~{\mdcolor{darkgreen}/*}{\mdcolor{darkgreen}~主界面底部栏~}{\mdcolor{darkgreen}*/}\\
~~~~tabBarComponent:~props~=\textgreater{}~\{\\
~~~~~~{\mdcolor{navy}return}~(\\
~~~~~~~~\textless{}Footer\textgreater{}\\
~~~~~~~~~~\textless{}FooterTab\textgreater{}\\
~~~~~~~~~~~~\textless{}Button\textgreater{}\{{\mdcolor{darkgreen}/*}{\mdcolor{darkgreen}~记账页面按钮~}{\mdcolor{darkgreen}*/}\}\textless{}/Button\textgreater{}\\
~~~~~~~~~~~~\textless{}Button\textgreater{}\{{\mdcolor{darkgreen}/*}{\mdcolor{darkgreen}~列表统计页面按钮~}{\mdcolor{darkgreen}*/}\}\textless{}/Button\textgreater{}\\
~~~~~~~~~~~~\textless{}Button\textgreater{}\{{\mdcolor{darkgreen}/*}{\mdcolor{darkgreen}~图表统计页面按钮~}{\mdcolor{darkgreen}*/}\}\textless{}/Button\textgreater{}\\
~~~~~~~~~~~~\textless{}Button\textgreater{}\{{\mdcolor{darkgreen}/*}{\mdcolor{darkgreen}~同步页面按钮~}{\mdcolor{darkgreen}*/}\}\textless{}/Button\textgreater{}\\
~~~~~~~~~~\textless{}/FooterTab\textgreater{}\\
~~~~~~~~\textless{}/Footer\textgreater{}\\
~~~~~~)\\
~~~~\}\\
~~\}\\
);%mdk
\end{mdpre}
%mdk-data-line={251}
\subsubsection{\mdline{251}4.3.2.\hspace*{0.5em}\mdline{251}记账模块}\label{section}%mdk%mdk

%mdk-data-line={253}
\noindent\mdline{253}记账模块提供账单列表界面和账单编辑界面。账单列表界面显示所有的账目,可以在添加或删除一条账目。
账单编辑界面显示一条具体的账目信息,包括交易发生的日期、时间,交易类型、项目、金额、种类,
以及相关图片和交易发生的地址。%mdk

%mdk-data-line={257}
\paragraph{\mdline{257}账单列表}\label{section}%mdk%mdk

%mdk-data-line={259}
\noindent\mdline{259}账单列表部分对应\mdline{259}\mdcode{src/views/Accounts.js}\mdline{259}文件。%mdk

%mdk-data-line={261}
\mdline{261}核心代码如下。\mdline{261}\mdcode{Container}\mdline{261},\mdline{261}\mdcode{Content}\mdline{261},\mdline{261}\mdcode{Button}\mdline{261}均是\mdline{261}\mdcode{native-base}\mdline{261}提供的组件。
\mdline{262}\mdcode{AccountList}\mdline{262}是自定义的列表界面。\mdline{262}\mdcode{Accounts}\mdline{262}会从记账模块的数据容器中获取账目列表数据\mdline{262}\mdcode{accounts}\mdline{262}
和一些可调用的函数如\mdline{263}\mdcode{onClickDel}\mdline{263},并将这些属性传递给子组件使用。%mdk
\begin{mdpre}%mdk
\noindent~~~~~~~~~~~~\textless{}Container\textgreater{}\\
~~~~~~~~~~~~~~~~\textless{}Content\textgreater{}\\
~~~~~~~~~~~~~~~~~~~~{\mdcolor{darkgreen}/*}{\mdcolor{darkgreen}~添加新条目按钮~}{\mdcolor{darkgreen}*/}\\
~~~~~~~~~~~~~~~~~~~~\textless{}Button~onPress=\{...\}\textgreater{}~~{\mdcolor{darkgreen}//~调用回调函数}\\
~~~~~~~~~~~~~~~~~~~~~~~~\textless{}Icon~.../\textgreater{}\\
~~~~~~~~~~~~~~~~~~~~~~~~\textless{}Text\textgreater{}添加\textless{}/Text\textgreater{}\\
~~~~~~~~~~~~~~~~~~~~\textless{}/Button\textgreater{}\\
~~~~~~~~~~~~~~~~~~~~{\mdcolor{darkgreen}/*}{\mdcolor{darkgreen}~账目列表~}{\mdcolor{darkgreen}*/}\\
~~~~~~~~~~~~~~~~~~~~\textless{}AccountList\\
~~~~~~~~~~~~~~~~~~~~~~~~accounts=\{accounts\}~{\mdcolor{darkgreen}//~传递账目数据}\\
~~~~~~~~~~~~~~~~~~~~~~~~onClickDel=\{...\}\\
~~~~~~~~~~~~~~~~~~~~~~~~onClickEdit=\{...\}\\
~~~~~~~~~~~~~~~~~~~~/\textgreater{}\\
~~~~~~~~~~~~~~~~\textless{}/Content\textgreater{}\\
~~~~~~~~~~~~\textless{}/Container\textgreater{}\\
\}%mdk
\end{mdpre}\noindent\mdline{284}定义\mdline{284}\mdcode{AccountList}\mdline{284}组件的代码如下,使用了\mdline{284}\mdcode{react-native}\mdline{284}提供的基本组件\mdline{284}\mdcode{FlatList}\mdline{284}作为列表界面。
\mdline{285}\mdcode{AccountItem}\mdline{285}是自定义的账单条目的视图。\mdline{285}\mdcode{AccountList}\mdline{285}从父组件处接收\mdline{285}\mdcode{accounts}\mdline{285}和回调函数,
将其传给\mdline{286}\mdcode{AccountItem}\mdline{286}。
\begin{mdpre}%mdk
\noindent~~~~~~~~{\mdcolor{darkgreen}/*}{\mdcolor{darkgreen}~对FlatList做封装,得到AccountList组件~}{\mdcolor{darkgreen}*/}\\
~~~~~~~~\textless{}FlatList\\
~~~~~~~~~~~~data=\{accounts\}\\
~~~~~~~~~~~~renderItem=\{(\{item,~index\})~=\textgreater{}~(\\
~~~~~~~~~~~~~~~~{\mdcolor{darkgreen}/*}{\mdcolor{darkgreen}~每一个账单条目~}{\mdcolor{darkgreen}*/}\\
~~~~~~~~~~~~~~~~\textless{}AccountItem\\
~~~~~~~~~~~~~~~~~~~~account=\{item\}~~~~~~~~{\mdcolor{darkgreen}//~传递账单条目}\\
~~~~~~~~~~~~~~~~~~~~index=\{index\}\\
~~~~~~~~~~~~~~~~~~~~onClickDel=\{onClickDel\}\\
~~~~~~~~~~~~~~~~~~~~onClickEdit=\{onClickEdit\}\\
~~~~~~~~~~~~~~~~/\textgreater{}\\
~~~~~~~~~~~~)\}\\
~~~~~~~~/\textgreater{}%mdk
\end{mdpre}\noindent\mdline{304}定义\mdline{304}\mdcode{AccountItem}\mdline{304}的代码如下,\mdline{304}\mdcode{AccountItem}\mdline{304}中定义了列表中应显示的简略账目信息,
以及左滑时出现的删除按钮。\mdline{305}\mdcode{AccountItem}\mdline{305}接收\mdline{305}\mdcode{accounts}\mdline{305}等属性,用于显示和交互。
\begin{mdpre}%mdk
\noindent~~~~~~~~{\mdcolor{darkgreen}/*}{\mdcolor{darkgreen}~将可滑动行做封装,得到AccountItem~}{\mdcolor{darkgreen}*/}\\
~~~~~~~~\textless{}SwipeRow\\
~~~~~~...\\
~~~~~~~~~~~~{\mdcolor{darkgreen}/*}{\mdcolor{darkgreen}~条目简略信息~}{\mdcolor{darkgreen}*/}\\
~~~~~~~~~~~~body=\{\\
~~~~~~~~~~~~~~~~\textless{}Button\\
~~~~~~~~~~...\\
~~~~~~~~~~~~~~~~~~~~onPress=\{()~=\textgreater{}~onClickEdit(index)\}\textgreater{}\\
~~~~~~~~~~~~~~~~~~~~\textless{}H2\textgreater{}\\
~~~~~~~~~~~~~~~~~~~~~~~~条目\{index\}:\\
~~~~~~~~~~~~~~~~~~~~\textless{}/H2\textgreater{}\\
~~~~~~~~~~~~~~~~~~~~\textless{}Text\textgreater{}\\
~~~~~~~~~~~~~~~~~~~~~~~~\{{\mdcolor{darkgreen}/*}{\mdcolor{darkgreen}~显示简略的账目信息~}{\mdcolor{darkgreen}*/}\}\\
~~~~~~~~~~~~~~~~~~~~\textless{}/Text\textgreater{}\\
~~~~~~~~~~~~~~~~\textless{}/Button\textgreater{}\\
~~~~~~~~~~~~\}\\
~~~~~~~~~~~~{\mdcolor{darkgreen}/*}{\mdcolor{darkgreen}~左滑时出现的删除按钮~}{\mdcolor{darkgreen}*/}\\
~~~~~~~~~~~~right=\{\\
~~~~~~~~~~~~~~~~\textless{}Button\\
~~~~~~~~~~...\\
~~~~~~~~~~~~~~~~~~~~onPress=\{()~=\textgreater{}~onClickDel(index)\}\textgreater{}\\
~~~~~~~~~~~~~~~~~~~~\textless{}Icon~active~name={\mdcolor{maroon}'}{\mdcolor{maroon}trash}{\mdcolor{maroon}'}~/\textgreater{}\\
~~~~~~~~~~~~~~~~\textless{}/Button\textgreater{}\\
~~~~~~~~~~~~\}\\
~~~~~~~~/\textgreater{}%mdk
\end{mdpre}
%mdk-data-line={335}
\paragraph{\mdline{335}账单编辑}\label{section}%mdk%mdk

%mdk-data-line={337}
\noindent\mdline{337}账单编辑部分对应\mdline{337}\mdcode{src/views/AccountEdit.js}\mdline{337}文件。%mdk

%mdk-data-line={339}
\mdline{339}核心代码如下,在一个表单里定义了各编辑组件。
\mdline{340}\mdcode{Form}\mdline{340}、\mdline{340}\mdcode{Item}\mdline{340}、\mdline{340}\mdcode{Input}\mdline{340}、\mdline{340}\mdcode{Label}\mdline{340}、\mdline{340}\mdcode{Picker}\mdline{340}、\mdline{340}\mdcode{Image}\mdline{340}等均为\mdline{340}\mdcode{react-native}\mdline{340}或\mdline{340}\mdcode{native-base}\mdline{340}
提供的基本组件。
\mdline{342}\mdcode{MyDatePicker}\mdline{342}、\mdline{342}\mdcode{MyTimePicker}\mdline{342}是自定义的选择日期和时间的组件。
\mdline{343}\mdcode{AccountEdit}\mdline{343}从数据容器中接收属性\mdline{343}\mdcode{accountData}\mdline{343}和一些回调函数。
\mdline{344}\mdcode{accountData}\mdline{344}表示当前修改的账单条目。%mdk
\begin{mdpre}%mdk
\noindent~~~~~~\textless{}Container\textgreater{}\\
~~~~~~~~\textless{}Content\textgreater{}\\
~~~~~~~~~~\textless{}Form\textgreater{}\\
~~~~~~~~~~~~~~~~~~~~~~~~{\mdcolor{darkgreen}/*}{\mdcolor{darkgreen}~选择日期的组件~}{\mdcolor{darkgreen}*/}\\
~~~~~~~~~~~~\textless{}Item~...\textgreater{}\\
~~~~~~~~~~~~~~\textless{}Label\textgreater{}日期\textless{}/Label\textgreater{}\\
~~~~~~~~~~~~~~\textless{}MyDatePicker~...\\
~~~~~~~~~~~~~~~~date=\{accountData.date\}~onChangeDate=\{onChangeDate\}~/\textgreater{}\\
~~~~~~~~~~~~\textless{}/Item\textgreater{}\\
~~~~~~~~~~~~~~~~~~~~~~~~...\\
~~~~~~~~~~~~~~~~~~~~~~~~{\mdcolor{darkgreen}/*}{\mdcolor{darkgreen}~选择账目类型的组件~}{\mdcolor{darkgreen}*/}\\
~~~~~~~~~~~~\textless{}Item~...\textgreater{}\\
~~~~~~~~~~~~~~\textless{}Label\textgreater{}账目类型\textless{}/Label\textgreater{}\\
~~~~~~~~~~~~~~\textless{}Picker\\
~~~~~~~~~~~~~~~~...\\
~~~~~~~~~~~~~~~~selectedValue=\{\\
~~~~~~~~~~~~~~~~~~accountData.isIncome~===~{\mdcolor{maroon}'}{\mdcolor{maroon}undefined}{\mdcolor{maroon}'}~?~{\mdcolor{navy}true}~:~accountData.isIncome\\
~~~~~~~~~~~~~~~~\}\\
~~~~~~~~~~~~~~~~onValueChange=\{(itemValue)~=\textgreater{}~onChangeType(itemValue)\}\textgreater{}\\
~~~~~~~~~~~~~~~~\textless{}Picker.Item~label={\mdcolor{maroon}'}{\mdcolor{maroon}收入}{\mdcolor{maroon}'}~value=\{{\mdcolor{navy}true}\}~/\textgreater{}\\
~~~~~~~~~~~~~~~~\textless{}Picker.Item~label={\mdcolor{maroon}'}{\mdcolor{maroon}支出}{\mdcolor{maroon}'}~value=\{{\mdcolor{navy}false}\}~/\textgreater{}\\
~~~~~~~~~~~~~~\textless{}/Picker\textgreater{}\\
~~~~~~~~~~~~\textless{}/Item\textgreater{}\\
~~~~~~~~~~~~~~~~~~~~~~~~{\mdcolor{darkgreen}/*}{\mdcolor{darkgreen}~选择消费种类(项目)的组件~}{\mdcolor{darkgreen}*/}\\
~~~~~~~~~~~~\textless{}Item~...\textgreater{}\\
~~~~~~~~~~~~~~\textless{}Label\textgreater{}消费种类\textless{}/Label\textgreater{}\\
~~~~~~~~~~~~~~\textless{}Picker\\
~~~~~~~~~~~~~~~~...\\
~~~~~~~~~~~~~~~~selectedValue=\{accountData.item\}\\
~~~~~~~~~~~~~~~~onValueChange=\{(itemValue)~=\textgreater{}~onChangeItem(itemValue)\}\textgreater{}\\
~~~~~~~~~~~~~~~~\textless{}Picker.Item~label={\mdcolor{maroon}'}{\mdcolor{maroon}购物}{\mdcolor{maroon}'}~value=\{{\mdcolor{maroon}'}{\mdcolor{maroon}购物}{\mdcolor{maroon}'}\}~/\textgreater{}\\
~~~~~~~~~~~~~~~~\textless{}Picker.Item~label={\mdcolor{maroon}'}{\mdcolor{maroon}餐饮}{\mdcolor{maroon}'}~value=\{{\mdcolor{maroon}'}{\mdcolor{maroon}餐饮}{\mdcolor{maroon}'}\}~/\textgreater{}\\
~~~~~~~~~~~~~~~~\textless{}Picker.Item~label={\mdcolor{maroon}'}{\mdcolor{maroon}服装}{\mdcolor{maroon}'}~value=\{{\mdcolor{maroon}'}{\mdcolor{maroon}服装}{\mdcolor{maroon}'}\}~/\textgreater{}\\
~~~~~~~~~~~~~~~~\textless{}Picker.Item~label={\mdcolor{maroon}'}{\mdcolor{maroon}生活}{\mdcolor{maroon}'}~value=\{{\mdcolor{maroon}'}{\mdcolor{maroon}生活}{\mdcolor{maroon}'}\}~/\textgreater{}\\
~~~~~~~~~~~~~~~~~~~~~~~~~~~~~~~~...\\
~~~~~~~~~~~~~~~~~~~~~~~~~~~~~~~~...\\
~~~~~~~~~~~~~~\textless{}/Picker\textgreater{}\\
~~~~~~~~~~~~\textless{}/Item\textgreater{}\\
~~~~~~~~~~~~~~~~~~~~~~~~...\\
~~~~~~~~~~~~~~~~~~~~~~~~...\\
~~~~~~~~~~\textless{}/Form\textgreater{}\\
~~~~~~~~\textless{}/Content\textgreater{}\\
~~~~~~\textless{}/Container\textgreater{}%mdk
\end{mdpre}
%mdk-data-line={392}
\paragraph{\mdline{392}账单数据的存储容器}\label{section}%mdk%mdk

%mdk-data-line={394}
\noindent\mdline{394}账单数据的保存应用了Redux框架,即程序中只有唯一的一份数据容器,且只能通过
回调函数间接操作数据内容,不允许在用户界面中对数据直接修改。%mdk

%mdk-data-line={397}
\mdline{397}账目信息对应的数据容器实现在\mdline{397}\mdcode{src/models/AccountReducer.js}\mdline{397}中。%mdk

%mdk-data-line={399}
\mdline{399}定义账单数据类型。%mdk
\begin{mdpre}%mdk
\noindent{\mdcolor{navy}class}~AccountData~\{\\
~~{\mdcolor{navy}constructor}(\{key\})~\{\\
~~~~{\mdcolor{navy}this}.key~=~key;~{\mdcolor{darkgreen}//~string}\\
~~~~{\mdcolor{navy}this}.date~=~moment({\mdcolor{navy}new}~Date()).format({\mdcolor{maroon}'}{\mdcolor{maroon}YYYY-MM-DD}{\mdcolor{maroon}'});~{\mdcolor{darkgreen}//~Date}\\
~~~~{\mdcolor{navy}this}.time~=~moment({\mdcolor{navy}new}~Date()).format({\mdcolor{maroon}"}{\mdcolor{maroon}LT}{\mdcolor{maroon}"});~{\mdcolor{darkgreen}//~Date}\\
~~~~{\mdcolor{navy}this}.isIncome~=~{\mdcolor{navy}false};~{\mdcolor{darkgreen}//~boolean:~is~income~or~expense}\\
~~~~{\mdcolor{navy}this}.amount~=~{\mdcolor{maroon}"}{\mdcolor{maroon}0}{\mdcolor{maroon}"}~{\mdcolor{darkgreen}//~string:~the~amount~of~money}\\
~~~~{\mdcolor{navy}this}.item~=~{\mdcolor{maroon}'}{\mdcolor{maroon}购物}{\mdcolor{maroon}'}~{\mdcolor{darkgreen}//~string:~on~what~item~the~transaction~is}\\
~~~~{\mdcolor{navy}this}.desc~=~undefined~{\mdcolor{darkgreen}//~string:~description~of~the~transaction}\\
~~~~{\mdcolor{navy}this}.imgPaths~=~{}[];~{\mdcolor{darkgreen}//~array(string)~paths~of~images}\\
~~~~{\mdcolor{navy}this}.position~=~undefined;~{\mdcolor{darkgreen}//~Position:~the~geolocation~where~the~transaction~happened}\\
~~\}\\
\}%mdk
\end{mdpre}\noindent\mdline{417}定义默认的容器数据。
\begin{mdpre}%mdk
\noindent{\mdcolor{navy}const}~INITIAL\_STATE~=~\{\\
~~next\_key:~{\mdcolor{purple}0},\\
~~accounts:~{}[],\\
~~index:~{\mdcolor{purple}0},\\
~~accountData:~{\mdcolor{navy}new}~AccountData(\{\}),\\
\};%mdk
\end{mdpre}\noindent\mdline{428}定义所有可能发生的数据操作行为。
\begin{mdpre}%mdk
\noindent{\mdcolor{navy}const}~accountsReducer~=~(state~=~INITIAL\_STATE,~action)~=\textgreater{}~\{\\
~~{\mdcolor{navy}switch}~(action.type)~\{\\
~~~~{\mdcolor{navy}case}~{\mdcolor{maroon}"}{\mdcolor{maroon}account\_add}{\mdcolor{maroon}"}:~{\mdcolor{navy}return}~handleAdd(state);\\
~~~~{\mdcolor{navy}case}~{\mdcolor{maroon}"}{\mdcolor{maroon}account\_del}{\mdcolor{maroon}"}:~{\mdcolor{navy}return}~handleDel(state,~action);\\
~~~~{\mdcolor{navy}case}~{\mdcolor{maroon}"}{\mdcolor{maroon}account\_edit}{\mdcolor{maroon}"}:~{\mdcolor{navy}return}~handleEdit(state,~action);\\
~~~~{\mdcolor{navy}case}~{\mdcolor{maroon}"}{\mdcolor{maroon}account\_save}{\mdcolor{maroon}"}:~{\mdcolor{navy}return}~handleSave(state);\\
~~~~{\mdcolor{navy}case}~{\mdcolor{maroon}"}{\mdcolor{maroon}account\_edit\_date}{\mdcolor{maroon}"}:~{\mdcolor{navy}return}~handleEditDate(state,~action);\\
~~~~{\mdcolor{navy}case}~{\mdcolor{maroon}"}{\mdcolor{maroon}account\_edit\_time}{\mdcolor{maroon}"}:~{\mdcolor{navy}return}~handleEditTime(state,~action);\\
~~~~{\mdcolor{navy}case}~{\mdcolor{maroon}"}{\mdcolor{maroon}account\_edit\_type}{\mdcolor{maroon}"}:~{\mdcolor{navy}return}~handleEditType(state,~action);\\
~~~~{\mdcolor{navy}case}~{\mdcolor{maroon}"}{\mdcolor{maroon}account\_edit\_amount}{\mdcolor{maroon}"}:~{\mdcolor{navy}return}~handleEditAmount(state,~action);\\
~~~~...\\
~~~~~~~~...\\
~~\}\\
~~{\mdcolor{navy}return}~state;\\
\}%mdk
\end{mdpre}
%mdk-data-line={448}
\subsubsection{\mdline{448}4.3.3.\hspace*{0.5em}\mdline{448}列表统计模块}\label{section}%mdk%mdk

%mdk-data-line={450}
\noindent\mdline{450}列表统计模块将每月的总收入和总支出显示给用户。为了方便地查看某一日的具体开支,还要能提供快捷的日期跳转功能。%mdk

%mdk-data-line={452}
\mdline{452}在\mdline{452}\mdcode{react-native-calendars}\mdline{452}中,提供了\mdline{452}\mdcode{Calendar}\mdline{452}组件以显示日历。日历具有点击左右箭头按钮切换月份、单击日期触发事件等功能。%mdk
\begin{mdpre}%mdk
\noindent~~~~~~~~~~~~~~~~\textless{}Calendar\\
~~~~~~~~~~~~~~~~~~~~onDayPress=\{(day)~=\textgreater{}\\
~~~~~~~~~~~~~~~~~~~~~~~~{\mdcolor{darkgreen}//~点击日期切换至消费详细}\\
~~~~~~~~~~~~~~~~~~~~~~~~onClick(day,~()~=\textgreater{}~\{\\
~~~~~~~~~~~~~~~~~~~~~~~~~~~~{\mdcolor{darkgreen}//~console.warn(day);}\\
~~~~~~~~~~~~~~~~~~~~~~~~~~~~navigation.navigate({\mdcolor{maroon}'}{\mdcolor{maroon}monthsDetail}{\mdcolor{maroon}'});\\
~~~~~~~~~~~~~~~~~~~~~~~~\})\\
~~~~~~~~~~~~~~~~~~~~\}\\
~~~~~~~~~~~~~~~~~~~~monthFormat~=~\{~{\mdcolor{maroon}'}{\mdcolor{maroon}yyyy年M月}{\mdcolor{maroon}'}~\}\\
~~~~~~~~~~~~~~~~~~~~onMonthChange~=~\{(month)~=\textgreater{}~\{\\
~~~~~~~~~~~~~~~~~~~~~~~~onChange(month);\\
~~~~~~~~~~~~~~~~~~~~~~~~onIncome(accounts);\\
~~~~~~~~~~~~~~~~~~~~~~~~onExpense(accounts);\\
~~~~~~~~~~~~~~~~~~~~\}\}\\
~~~~~~~~~~~~~~~~/\textgreater{}%mdk
\end{mdpre}\noindent\mdline{472}同时,为了实现UI上的复用,避免多个模块风格不一致,
在\mdline{473}\mdcode{Months}\mdline{473}类型被封装为用户可见的\mdline{473}\mdcode{MonthsView}\mdline{473}前,我们使用\mdline{473}\mdcode{react-navigation}\mdline{473}中的功能指定其“标题栏”的外观。
\begin{mdpre}%mdk
\noindent~~{\mdcolor{navy}static}~navigationOptions(\{navigation\})~\{\\
~~~~~~~~{\mdcolor{navy}return}~\{\\
~~~~~~~~~~~~title:~{\mdcolor{maroon}'}{\mdcolor{maroon}Months}{\mdcolor{maroon}'},\\
~~~~~~~~~~~~header:~(\\
~~~~~~~~~~~~~~~~\textless{}Header\textgreater{}\\
~~~~~~~~~~~~~~~~~~~~\textless{}Left~/\textgreater{}\\
~~~~~~~~~~~~~~~~~~~~\textless{}Body\textgreater{}\\
~~~~~~~~~~~~~~~~~~~~~~~~\textless{}Title\textgreater{}月份\textless{}/Title\textgreater{}\\
~~~~~~~~~~~~~~~~~~~~\textless{}/Body\textgreater{}\\
~~~~~~~~~~~~~~~~~~~~\textless{}Right~/\textgreater{}\\
~~~~~~~~~~~~~~~~\textless{}/Header\textgreater{}\\
~~~~~~~~~~~~)\\
~~~~~~~~\};\\
~~\}%mdk
\end{mdpre}\noindent\mdline{492}在列表统计模块内部的触发事件可分为以下几类。

%mdk-data-line={494}
\begin{itemize}[noitemsep,topsep=\mdcompacttopsep]%mdk

%mdk-data-line={494}
\item\mdline{494}点击日历上的具体日期

%mdk-data-line={495}
\begin{itemize}[noitemsep,topsep=\mdcompacttopsep]%mdk

%mdk-data-line={495}
\item\mdline{495}此时更新状态中的年、月、日,对应到该日期,通过年月日筛选出该日期下的账目%mdk

%mdk-data-line={496}
\item\mdline{496}触发从当前视图\mdline{496}\mdcode{MonthsView\_}\mdline{496}转移到\mdline{496}\mdcode{MonthsDetailView}\mdline{496}的事件%mdk
%mdk
\end{itemize}%mdk%mdk

%mdk-data-line={497}
\item\mdline{497}点击日历的左右切换月份按钮

%mdk-data-line={498}
\begin{itemize}[noitemsep,topsep=\mdcompacttopsep]%mdk

%mdk-data-line={498}
\item\mdline{498}更新状态中的月%mdk

%mdk-data-line={499}
\item\mdline{499}更新当前月的总收入和总支出%mdk
%mdk
\end{itemize}%mdk%mdk
%mdk
\end{itemize}%mdk
\begin{mdpre}%mdk
\noindent~~{\mdcolor{navy}const}~mapDispatchToProps~=~(dispatch)~=\textgreater{}~(\{\\
~~~~~~onClick:~(day,~callBack)~=\textgreater{}~\{\\
~~~~~~~~~~dispatch(\{~type:~{\mdcolor{maroon}'}{\mdcolor{maroon}year\_select}{\mdcolor{maroon}'},~year:~day.year~\});\\
~~~~~~~~~~dispatch(\{~type:~{\mdcolor{maroon}'}{\mdcolor{maroon}month\_select}{\mdcolor{maroon}'},~month:~day.month~\});\\
~~~~~~~~~~dispatch(\{~type:~{\mdcolor{maroon}'}{\mdcolor{maroon}day\_select}{\mdcolor{maroon}'},~day:~day.day~\});\\
~~~~~~~~~~dispatch(\{~type:~{\mdcolor{maroon}'}{\mdcolor{maroon}month\_watch}{\mdcolor{maroon}'},~callBack:~callBack~\})\\
~~~~~~~~~~console.log({\mdcolor{maroon}'}{\mdcolor{maroon}WATCH}{\mdcolor{maroon}'});\\
~~~~~~\},\\
~~~~~~onChange:~(month)~=\textgreater{}~\{\\
~~~~~~~~~~dispatch(\{~type:~{\mdcolor{maroon}'}{\mdcolor{maroon}month\_change}{\mdcolor{maroon}'},~month:~month.month~\});\\
~~~~~~\},\\
~~~~~~onIncome:~(accounts)~=\textgreater{}~\{\\
~~~~~~~~~~dispatch(\{~type:~{\mdcolor{maroon}'}{\mdcolor{maroon}month\_income}{\mdcolor{maroon}'},~accounts:~accounts~\});\\
~~~~~~\},\\
~~~~~~onExpense:~(accounts)~=\textgreater{}~\{\\
~~~~~~~~~~dispatch(\{~type:~{\mdcolor{maroon}'}{\mdcolor{maroon}month\_expense}{\mdcolor{maroon}'},~accounts:~accounts~\});\\
~~~~~~\},\\
~~\});%mdk
\end{mdpre}\noindent\mdline{522}点击某一具体日期后,查看具体账目所涉及的逻辑在\mdline{522}\mdcode{MonthsDetail.js}\mdline{522}中。
在\mdline{523}\mdcode{AccountData}\mdline{523}这一对象类型数据中,\mdline{523}\mdcode{date}\mdline{523}域为形如\mdline{523}\mdcode{"YYYY-MM-DD"}\mdline{523}的字符串。
需要注意的是,诸如\mdline{524}\mdcode{6}\mdline{524}、\mdline{524}\mdcode{7}\mdline{524}月在字符串中也会变为\mdline{524}\mdcode{06}\mdline{524}、\mdline{524}\mdcode{07}\mdline{524}。据此可以写出筛选当日账目并按具体时间排序显示的逻辑。
\begin{mdpre}%mdk
\noindent~~~~~~~~{\mdcolor{navy}var}~res~=~{}[];\\
~~~~~~~~{\mdcolor{navy}for}~({\mdcolor{navy}var}~i~=~{\mdcolor{purple}0};~i~\textless{}~accounts.length;~++~i)~\{\\
~~~~~~~~~~~~{\mdcolor{navy}if}~((month~\textless{}~{\mdcolor{purple}10}~\&\&~year~+~{\mdcolor{maroon}"}{\mdcolor{maroon}-0}{\mdcolor{maroon}"}~+~month~+~{\mdcolor{maroon}"}{\mdcolor{maroon}-}{\mdcolor{maroon}"}~+~day~==~accounts{}[i].date)~\textbar{}\textbar{}\\
~~~~~~~~~~~~(month~\textgreater{}=~{\mdcolor{purple}10}~\&\&~year~+~{\mdcolor{maroon}"}{\mdcolor{maroon}-}{\mdcolor{maroon}"}~+~month~+~{\mdcolor{maroon}"}{\mdcolor{maroon}-}{\mdcolor{maroon}"}~+~day~==~accounts{}[i].date~))~\{\\
~~~~~~~~~~~~~~~~res.push(accounts{}[i]);\\
~~~~~~~~~~~~\}\\
~~~~~~~~\}\\
~~~~~~~~res.sort({\mdcolor{navy}function}(a,~b)\{{\mdcolor{navy}return}~a.time~-~b.time\});%mdk
\end{mdpre}
%mdk-data-line={537}
\subsubsection{\mdline{537}4.3.4.\hspace*{0.5em}\mdline{537}图表统计模块}\label{section}%mdk%mdk

%mdk-data-line={539}
\noindent\mdline{539}我们以饼状图展现每月中,各收入/支出类别在当月总收入/支出的占比。%mdk

%mdk-data-line={541}
\mdline{541}在\mdline{541}\mdcode{react-native-chart-kit}\mdline{541}中,\mdline{541}\mdcode{PieChart}\mdline{541}组件能够满足我们的需求。
在\mdline{542}\mdcode{StatisticsReducer.js}\mdline{542}中,保存的状态用于图表统计模块的制图。
与列表统计模块不同的是,图表统计模块中使用\mdline{543}\mdcode{CalendarList}\mdline{543},无需点按左右箭头按钮切换月份,而是直接左右滑动屏幕即可。%mdk

%mdk-data-line={545}
\mdline{545}状态中,\mdline{545}\mdcode{categories}\mdline{545}中参与统计的收支类型需要与之前记账模块中内置的类型保持一致,\mdline{545}\mdcode{income}\mdline{545}和\mdline{545}\mdcode{expense}\mdline{545}两个数据分别用于收入图和支出图。
但因为所有数据全部为\mdline{546}\mdcode{0}\mdline{546}时,在渲染时会报错,因此我们暂时将“其他”项这里修改为非\mdline{546}\mdcode{0}\mdline{546}。
这只是权宜之计,实际统计后往往会被覆盖掉。%mdk
\begin{mdpre}%mdk
\noindent~~{\mdcolor{navy}const}~INITIAL\_STATE~=~\{\\
~~~~month:~moment().format({\mdcolor{maroon}'}{\mdcolor{maroon}YYYY-MM}{\mdcolor{maroon}'}),\\
~~~~categories:~{}[\\
~~~~~~\{~name:~{\mdcolor{maroon}'}{\mdcolor{maroon}购物}{\mdcolor{maroon}'},~income:~{\mdcolor{purple}0},~expense:~{\mdcolor{purple}0},~\},\\
~~~~~~\{~name:~{\mdcolor{maroon}'}{\mdcolor{maroon}餐饮}{\mdcolor{maroon}'},~income:~{\mdcolor{purple}0},~expense:~{\mdcolor{purple}0},~\},\\
~~~~~~\{~name:~{\mdcolor{maroon}'}{\mdcolor{maroon}服装}{\mdcolor{maroon}'},~income:~{\mdcolor{purple}0},~expense:~{\mdcolor{purple}0},~\},\\
~~~~~~\{~name:~{\mdcolor{maroon}'}{\mdcolor{maroon}生活}{\mdcolor{maroon}'},~income:~{\mdcolor{purple}0},~expense:~{\mdcolor{purple}0},~\},\\
~~~~~~\{~name:~{\mdcolor{maroon}'}{\mdcolor{maroon}教育}{\mdcolor{maroon}'},~income:~{\mdcolor{purple}0},~expense:~{\mdcolor{purple}0},~\},\\
~~~~~~\{~name:~{\mdcolor{maroon}'}{\mdcolor{maroon}娱乐}{\mdcolor{maroon}'},~income:~{\mdcolor{purple}0},~expense:~{\mdcolor{purple}0},~\},\\
~~~~~~\{~name:~{\mdcolor{maroon}'}{\mdcolor{maroon}出行}{\mdcolor{maroon}'},~income:~{\mdcolor{purple}0},~expense:~{\mdcolor{purple}0},~\},\\
~~~~~~\{~name:~{\mdcolor{maroon}'}{\mdcolor{maroon}医疗}{\mdcolor{maroon}'},~income:~{\mdcolor{purple}0},~expense:~{\mdcolor{purple}0},~\},\\
~~~~~~\{~name:~{\mdcolor{maroon}'}{\mdcolor{maroon}投资}{\mdcolor{maroon}'},~income:~{\mdcolor{purple}0},~expense:~{\mdcolor{purple}0},~\},\\
~~~~~~\{~name:~{\mdcolor{maroon}'}{\mdcolor{maroon}其他}{\mdcolor{maroon}'},~income:~{\mdcolor{purple}1},~expense:~{\mdcolor{purple}1},~\},\\
~~~~],\\
~~~~yearData:~{}[],\\
~~\};%mdk
\end{mdpre}\noindent\mdline{568}我们分别对总收入占比和总支出占比进行饼状图制图。
\begin{mdpre}%mdk
\noindent~~~~\textless{}H3~style=\{\{paddingLeft:~{\mdcolor{purple}50},\}\}\textgreater{}月收入统计\textless{}/H3\textgreater{}\\
~~~~\textless{}PieChart\\
~~~~~~~~~~~~...\\
~~~~~~data=\{ctg\_income\}\\
~~~~~~accessor={\mdcolor{maroon}'}{\mdcolor{maroon}population}{\mdcolor{maroon}'}\\
~~~~~~backgroundColor={\mdcolor{maroon}'}{\mdcolor{maroon}transparent}{\mdcolor{maroon}'}\\
~~~~/\textgreater{}\\
\\
~~~~\textless{}H3~...\textgreater{}月支出统计\textless{}/H3\textgreater{}\\
~~~~\textless{}PieChart\\
~~~~~~~~~~~~...\\
~~~~~~data=\{ctg\_expense\}\\
~~~~~~accessor={\mdcolor{maroon}'}{\mdcolor{maroon}population}{\mdcolor{maroon}'}\\
~~~~~~backgroundColor={\mdcolor{maroon}'}{\mdcolor{maroon}transparent}{\mdcolor{maroon}'}\\
~~~~/\textgreater{}%mdk
\end{mdpre}\noindent\mdline{588}最后,在切换月份使得\mdline{588}\mdcode{month}\mdline{588}状态刷新时,我们也要能即时刷新图表显示。这一步的逻辑与列表统计模块类似。
首先更新\mdline{589}\mdcode{month}\mdline{589}属性,然后用更新过的时间属性筛选出对应时间范围内的账目,用它们更新\mdline{589}\mdcode{categories}\mdline{589}中的各条目的收入支出。

%mdk-data-line={591}
\subsubsection{\mdline{591}4.3.5.\hspace*{0.5em}\mdline{591}云端同步模块}\label{section}%mdk%mdk

%mdk-data-line={593}
\noindent\mdline{593}云端同步功能我们尚未实现,将会在期末大作业中加入。%mdk

%mdk-data-line={596}
\section{\mdline{596}5.\hspace*{0.5em}\mdline{596}系统可能的拓展}\label{section}%mdk%mdk

%mdk-data-line={598}
\begin{enumerate}%mdk

%mdk-data-line={598}
\item{}
%mdk-data-line={598}
\mdline{598}云端同步功能:支持从本地向云端同步数据或是从云端像本地同步数据。%mdk%mdk

%mdk-data-line={600}
\item{}
%mdk-data-line={600}
\mdline{600}导出功能:支持将选定日期范围内的账目导出至\mdline{600}\mdcode{.csv}\mdline{600}或\mdline{600}\mdcode{.xls}\mdline{600}格式的文件中,并能方便转移到其他平台。%mdk%mdk

%mdk-data-line={602}
\item{}
%mdk-data-line={602}
\mdline{602}更详细的统计:支持更多维度的统计数据,如年消费项目占比,消费额变化曲线等。%mdk%mdk

%mdk-data-line={604}
\item{}
%mdk-data-line={604}
\mdline{604}设置功能:制定多套显示模式、UI界面等,让用户自定义自己的记账软件。%mdk%mdk
%mdk
\end{enumerate}%mdk

%mdk-data-line={606}
\section{\mdline{606}6.\hspace*{0.5em}\mdline{606}总结体会}\label{section}%mdk%mdk

%mdk-data-line={608}
\noindent\mdline{608}这次Android作业难度,仅从代码量而言,相较于IOS作业有大提升,对于期末阶段的学生来说确实
是不小的压力,但是相应的收获也非常丰富。%mdk

%mdk-data-line={611}
\mdline{611}首先锻炼的是压力下高效产出代码的能力。我们小组两人在一周的时间内完成了这个程序的开发,对于
相对还缺乏开发经验的学生来说是个比较令人满意的结果。在这一学期的工程训练之后,不少同学对
项目的开发已经有了一定程度的熟悉和了解,这一点对于未来融入工作环境非常有益。%mdk

%mdk-data-line={615}
\mdline{615}其次是学到了一些新技术。我们在开发过程中对React和Redux框架越来越熟悉。
在实现各种功能时,我们接触了不少React,React-Native框架的组件,深刻地认识到了开源社区的力量。
开源社区相互友好协助的氛围更有利于新技术的传播和改进。%mdk%mdk


\end{document}
